%===========================================
% Preamble

\documentclass[11pt]{article}


%**********
%Dependencies

%**********
%Dependencies
%\usepackage[left]{lineno}
\usepackage{titlesec}
\usepackage{amsmath}
\usepackage{amsfonts}
\usepackage{amssymb}
%\usepackage[utf8]{inputenc}
\usepackage{color,soul}
\usepackage[sc]{mathpazo} %Like Palatino with extensive math support
\usepackage{fullpage}
\usepackage[authoryear,sectionbib,sort]{natbib}
\linespread{1}
\usepackage[utf8]{inputenc}
\usepackage{lineno}
\usepackage[hidelinks]{hyperref}


% New commands: fonts
\newcommand{\code}{\fontfamily{pcr}\selectfont}
\newcommand*\chem[1]{\ensuremath{\mathrm{#1}}}

\renewcommand\textbullet{\ensuremath{\bullet}}
%===========================================
% Title page


% \title{Cover Letter}
% \author{Colin Olito}
% \date{\today}


\begin{document}
%\maketitle
%\newpage{}


%===========================================
% Document


\section*{}
\noindent To the Ed., AE, \& Editorial Board
\bigskip

\noindent Please consider this revision of our manuscript (MS \#60734), titled "The demographic costs of sexually antagonistic selection in partially selfing populations" for publication in \textit{The American Naturalist}. 
\bigskip

\noindent First of all, we want to thank the Ed., AE, and EiC for their thoughtful and forthright responses (and apologies) regarding the editorial delay on our manuscript. We understand completely that pandemic-related personal issues have been and continue to be extremely disruptive in these difficult times, and that editorial duties for our journals is a volunteer service. To our AE, Dr.~Orive, we sincerely hope that you and your family are well, and we truly appreciate the time and effort you took to evaluate our paper when you clearly had more imporant issues to deal with.
\bigskip

\noindent We received very helpful and constructive comments on our first submission and have now made a variety of major revisions to the MS in order to address them. First, we have made significant changes to our introduction to develop conceptual links between our model and previous work on evolutionarily explicit structured models and empirical research on sexually antagonistic selection. Second, we have tried to lessen the burden on non-theoretical readers by shifting development of the general model to the Supplementary Material, by following R2's many helpful suggestions regarding the presentation of the 2-stage model, and by including a new explanatory figure to help clarify the genetic and demographic outcomes from our model. Third, we have revised several figures to show population intrinsic growth rates ($\lambda$) rather than extinction boundaries. Finally, we have addressed the impacts of density dependence by ({\itshape i}) developing a new version of our model which includes simple density-dependence and reanalyzing the {\itshape M. guttatus} example; and ({\itshape ii}) expanding our discussion of density-dependence to address these additions. Overall, these various additions have lengthened the manuscript \textcolor{blue}{by about a half-page}, but we feel that they have greatly improved the paper. 
\bigskip

\noindent We give detailed responses to each of the editors' and reviewers' comments below. For clarity, we use bold face for all line and figure references to the revised ms in our responses. We have also used blue text in the revised manuscript file to help reviewers identify the key additions we have made.
\bigskip

\noindent Thank you again for your efforts and consideration.
\bigskip

\noindent Sincerely,

\noindent The Authors \\
%\noindent Colin Olito \& Charlotte DeVries \\
\noindent\makebox[\linewidth]{\rule{\textwidth}{0.4pt}}
\smallskip


%%%%%%%%%%%%%%%%%%%%
\section*{Decision letter with our responses}

\subsection*{Responses to Editor Erol Ak\c{c}ay}


I must first apologize for the time it took to return this decision, and thank you for your patience. Even in the best of times, unexpected events and circumstances cause delays for reviewers and editors, and many of us are still operating under the continuing strain of the pandemic. In this case, I must also apologize to have added to the delay personally, which is because of a backlog that built up due to disruptions that happened in the last couple of months that ordinarily might not have led to huge issues, but after a year and a half of the pandemic seem to snowball out of control.

\begin{quote}
	{\itshape Thank you, again, for your candid and forthright apology and explanation. We truly appreciate it, and sympathize with the growing challenges that seem to creep up on us all during these difficult times.}
\end{quote}


The two reviewers, Dr. Orive, and myself all read your paper with great interest. We are all agreed that your model is well-constructed, and well-explained, and your conclusions are clearly presented. The insight that sexually antagonistic genetic variation might depress population demographic rates and affect population viability is a valuable one. We also appreciate that you use empirical data from a case study to parameterize your model, and agree that this is very useful to judge the real world relevance of your calculations. As such, there is good potential for this manuscript to be a great American Naturalist paper.

\begin{quote}
	{\itshape This is very encouraging indeed, and we certainly hope that our revisions can successfully fulfill the potential you see in our paper.}
\end{quote}

As you will see in Dr. Orive’s excellent summary of the reviews, one issue that we feel you need to address is to make the paper more suitable for the general audience of the American Naturalist. To my mind, this has two parts: First, I think you can reduce the amount of space you devote to the general model in the main text (this can go into the SI as a generalization and also as a reference for the later three-stage model for Mimulus), as most of the biological conclusions actually come from the two-stage model which requires less notation and the construction of which is more easily explained. Although I personally appreciated the construction of the general model, I think for our non-theoretical readership, the cost/benefit ratio of reading through the general model (which is not used for the numerical calculations) is too great. 

\begin{quote}
	{\itshape We agree that shifting the focus from deriving the general model to the specific two stage example will make the paper more accessible and shorten the Methods. We have now moved the development of the general model to the Online Supplementary Material (Supplement 1), and restructured the Methods in the main paper accordingly. }
\end{quote}

Second, like Reviewer 1 and Dr. Orive, I do think you should better connect your model to the broader topics of evolution in structured populations and sexual antagonism. That would give your paper a broader impact in terms of readership.

\begin{quote}
	{\itshape We agree entirely. Please see our detailed responses to Dr.~Orive and R1 on this count.}
\end{quote}


For my own part, I do have one additional major comment, which concerns how you decide on demographic viability. As you are admirably explicit in the model construction, this is a linear projection model with no density dependence in demographic rates. As such, the population growth or decline you infer from a given parameter set is necessarily "local," i.e., it applies around a particular population density where the base demographic rates are measured or modeled. To my mind, this is a potentially major gap between model calculations and what one can say about viability of actual populations in the wild, which will have density dependence and therefore will face varying demographic rates with density. For example you can have fertility rates that give a growth rate less than 1 at some densities, but as density becomes low, mortality might decrease (or fertility increase) to bring growth rates back to one. (That is of course the assumption of most eco-evolutionary theory such as adaptive dynamics: some sort of density dependence is assumed to keep populations at demographic equilibrium while new traits invade.) I understand that incorporating such non-linear, density dependent feedbacks would make the model complicated, but I also think that their lack in the current model does dampen your conclusions, in the sense that they don’t necessarily mean that populations would not be able to persist if their growth rate at some undefined density becomes less than one. At the very least, this is an issue that deserves acknowledgment and fairly extensive discussion because it directly affects your major conclusion. I was disappointed that the issue only came up once in passing in the discussion. Ideally, you could present some numerical simulations with density dependence. I sorta suspect that what you might get (at least for some cases) is depressed population densities rather than outright population inviability. I don’t know how difficult these simulations would be, but at least some numerical exploration might not be too much out of the way.

\begin{quote}
	{\itshape This is a very good and totally valid point. To be honest, we believe that a separate paper is necessary to deal adequately with the many consequences of density dependence for our model predictions. However, we also agree that sidestepping this issue completely isn't good enough. In the end, we have made two main changes to address density dependence. First, have added a new {\bf Appendix B} where we briefly explain how density dependence can be incorporated into the model, and reanalyze our {\itshape M. guttatus} example with a simple DD version of the model. Second, we have expanded our results and discussion to more fully address the implications of DD, with reference to our modified figures and the new Appendix ({\bf L.XXX--XXX}).}
\end{quote}


Related to that, and without doing more modeling, one quantity of interest you can present is the actual growth rate, and not just the region of viability/inviability. In a world where we acknowledge density dependence as a possibility, how far above or below one growth rate is can be informative about whether you actually expect extinction or depressed densities.

\begin{quote}
	{\itshape This is great suggestion. We have now included a new illustrative figure ({\bf fig.~1}), and revised versions of figs. 1 \& 4 (now {\bf figs.~S1 \& 5}) to show heatmaps of absolute values for predicted $\lambda$ instead of the extinction boundaries. However, one unfortunate consequence of presenting $\lambda$ instead of the extinction boundaries is that it is no longer possible to illustrate multiple values of maximum fertility ($f$) on each figure panel. In particular, our old fig. 1 now requires 18 panels to illustrate $\lambda$ for the same set of parameter combinations. We felt that this resulted in an unnecessarily complicated figure, especially since fig. 1's main purpose was to illustrate the general behaviour of the model for different fertility values. For simplicity, we have decided to retain our original version of fig.1 (now {\bf fig. 2}) in the main text and include the modified version showing $\lambda$ as a supplementary figure (now {\bf fig.~XXX}); but will swap this figure if the editor/reviewers feel differently. However, we agree that the {\itshape M. guttatus} example was greatly improved by showing $\lambda$, and we now present the modified version of fig. 4 (now {\bf fig. 5}) in the main text}.
\end{quote}


In addition to these general points, both reviewers and Dr. Orive have numerous helpful suggestions and corrections that will no doubt be useful while you revise the manuscript, and I urge you to pay close attention to them.
\bigskip


\noindent xxxxxxxxxxxxxxxxxxxxxxxxxxxxxxxxxxxxxxxxxxxxxxxxx

\section*{Responses to Associate Editor Maria Orive}

Both reviewers found the work interesting and potentially to be of strong interest to the readership of AmNat, and I agree. However, Reviewer 1 brings up an important point regarding the framing of the work and placing it within a broader context of the literature on evolutionarily explicit structured models, which I think is important for the authors to do in order to make the paper accessible to and of interest for the overall readership of AmNat. There are other points that the reviewers bring up that are important to consider, but I think this one needs to be addressed carefully.

\begin{quote}
	{\itshape We agree that in our efforts to appeal to both demographers and evolutionary biologists studying sexual conflict, we ended up insufficiently discussing the relevant literature on both counts (but especially regarding evolutionarily explicit structured models). We have made a variety of changes to our introduction to improve this ({\bf L.XX--XX, L.XX-XX}). See also our specific responses to Reviewer 1. }
\end{quote}


Reviewer 2 has some very nice suggestions regarding ideas for simplifying and clarifying the model description.

\begin{quote}
	{\itshape Agreed. Please see our sepecific responses to Reviewer 2 below.}
\end{quote}

In addition to the clarifications/edits and other points raised by the reviewers, I include a handful of minor edits/typos here of my own:
\bigskip

\noindent (1) Abstract, perhaps add a very general and short explanation of the term sexually antagonistic polymorphism.
\begin{quote}
	{\itshape ...}
\end{quote}

\noindent (2) Line 165; "fertility"
\begin{quote}
	{\itshape Fixed.}
\end{quote}

\noindent (3) Table 1, definition of inbreeding depression terms; it would be clearer to define each of the 4 terms separately, "Inbreeding depression terms for ovule viability, juvenile survival, adult survival, and transition rates respectively"
\begin{quote}
	{\itshape Fixed}
\end{quote}

\noindent (4) Figure 1; I generally prefer axes labels to have clear meanings, rather than just symbols – here, the vertical and horizontal axes give the female (vertical) vs. male (horizontal) selection on reproductive function.
\begin{quote}
	{\itshape We have changed our axes labels to be more descriptive }
\end{quote}

\noindent (5) Lines 305, 306, and 478; the use of the term "fixates" rather than "fixes" is not standard, I would change to "fixes" in all cases.
\begin{quote}
	{\itshape Fixed, or rephrased as "goes to fixation".}
\end{quote}

\noindent (6) Figure 4; same comment as for Figure 1.
\begin{quote}
	{\itshape Fixed.}
\end{quote}

Finally, I thank the authors for their patience; the delay in receiving these reviews lies not with the reviewers, but with myself, due to some personal and family issues that have recently arisen.

\begin{quote}
	{\itshape Thank you again for your efforts and consideration in spite of these challenges. We really appreciate it.}
\end{quote}


\noindent xxxxxxxxxxxxxxxxxxxxxxxxxxxxxxxxxxxxxxxxxxxxxxxxx

\subsection*{Reviewer \#1:}

The manuscript reports the derivation, analysis, and example application of an extended frequency-dependent matrix model to explore links between sexual antagonistic selection, population persistence, and the maintenance of heritable variation. The paper is very well written and is straightforward to follow. The derivations are correct, although I did not try to derive all the equations in the appendices. My one comment is how might environmental stochasticity in the rate of selfing and outcrossing impact general conclusions? 
\begin{quote}
	{\itshape We now briefly discuss this possibility in the Discussion ({\bf L.586--593}).}
\end{quote}


Other than that, I have no concerns about the science that is reported. My concerns rate the very narrow focus of the paper as written.

The manuscript adds to a growing and important literature where eco-evolutionary feedbacks are examined in structured models, some of which incorporate evolution explicitly. It also contributes nicely to a body of empirical and theoretical work on sexually antagonistic selection.

\begin{quote}
	{\itshape Thank you for these encouraging words.}
\end{quote}

As written the manuscript will not appeal to the broad readership of the American Naturalist because the weakness of the manuscript is the author makes little attempt to fit the work into the broader relevant literature. Instead, the paper has been written for a small audience of experts interested in the demographic consequences of sexually antagonistic selection. As written, the work is suited to a journal such as Journal of Theoretical Biology. This is disappointing, and the authors do themselves no favors by writing for such a niche audience. The manuscript could be well-suited to the American Naturalist if the authors were to engage with the broader literature. This can be done by extending the introduction to link this work to the wider literature on evolutionarily explicit structured models, and to the large empirical literature on sexually antagonistic selection. If they are able to do this appropriately, I believe this paper would make a strong contribution to the American Naturalist.

\begin{quote}
	{\itshape This is a really important comment and we thank you for pointing out this major shortcoming. One of our main goals while crafting the first draft of the manuscript was to find a middle ground that would appeal to both demographers and people studying sexual conflict. It is clear that we didn't quite achieve that goal. Rather, by making compromises in the introduction we ended up lacking scholarship on both sides. To address your main concern, we have extended the Introduction to more directly connect with the broader literature on evolutionarily explicit structured models (see {\bf L.66-83}) and now also provide a more thorough link with the empirical sexually antagonistic selection literature (see {\bf L.42--49}). We hope these changes succeed in broadening the appeal of our manuscript to the wider readership of AmNat.}
\end{quote}

\noindent xxxxxxxxxxxxxxxxxxxxxxxxxxxxxxxxxxxxxxxxxxxxxxxxx


\subsection*{Reviewer \#2:}

I have read, with great interest, this manuscript analyzing the effects of sexually antagonistic selection in hermaphrodites at both the population genetics and population dynamics levels. As a representative (I guess) of the general readership of AmNat and also as a population dynamics specialist, I appreciated the description of the model, its parametrization, the way it is constructed and analyzed. Not being a population genetics specialist, it is harder for me to comment on that side of the interpretations, apart from generic comments and in the light of the model’s characteristics. Overall, my take is that this paper is interesting, well written, the model well built and the ideas quite novel; and that it therefore corresponds to what readers except to find in AmNat both in terms of content and quality.

\begin{quote}
	{\itshape Thank you for the encouraging assessment.}
\end{quote}

I have one main comment:
-- Discussion: Conclusions are drawn before limitations of the model are mentioned. However, it seems to me that it would be very important to discuss the effect of those limitations on the results themselves and therefore discuss the generality of the results. For instance, what is the potential effect of female dominance in the marriage function on the results and in particular on conclusion of lines 477-480.

\begin{quote}
	{\itshape While we feel that there is strong precedence for discussing the key results of a study before diving into the various caveats that come with the analysis, the point is well taken that we need to clearly direct readers to the places in our paper where these limitations are discussed. In our Methods section, we are now careful to point readers to the various Appendixes, Supplements, and places in the Discussion where these limitations are dealt with (e.g., see {\bf 100-105; 112--117; 149--155; 210; fig.~2 caption}), and we have added more explicit discussion of several of our key assumptions in the discussion ({\bf e.g., L.556--574; 594--601})}
\end{quote}

\noindent Apart from that, I have only minor comments, that I list below (roughly in decreasing order of importance):
\bigskip 

\noindent -- Model: lines 114-195
\begin{itemize}
	\item The matrices H are mentioned before they are introduced. (Therefore, the parenthesis in lines 123-124 cannot be understood by the reader). And then are only defined in the following section	
	\begin{itemize}
		\item It would make sense to reorder the model presentation by presenting the building blocks (arguably U, F and H matrices, but I'll come back to H in the next point), what they do, what they are for; then provide some properties (F independent from selfing/crossing, H from stage, etc ...) and then go into the details of their construction
	\end{itemize}
	\item The main issue I have here, is that H is introduced, but then disappears from the output (see equations 12 and 13 for instance that use H but it does not appear, or the general output eq 9)
	\begin{itemize}
		\item This suggests combining F and H directly into an object.
	\end{itemize}
\end{itemize}

\begin{itemize}
	\item More generally, the matrix formulations of the model may be a bit hard to read for the general reader of AmNat. Getting rid of H (or to the contrary, using it all the way, thus simplifying eq 12 and 13 for instance) and not giving a name to matrix W (which is just a bunch of 0s 1s and 1/2s) are steps that can make it easier for the reader to follow.
\begin{quote}
	{\itshape Response to the above 5 points: As per the reviewers (and the editors) suggestion, we have moved a lot of the derivations into the supplementary materials, and now no longer use H or W in the main text. In doing so, we hope to shift the focus from mathematical derivations, which are probably unhelpful to many readers, towards explaining the end result instead. }
\end{quote}

	\item It is quite confusing that HX is written as a constant (eq 7) whilst it depends on the frequencies and that HS is written as dependent (eq6) whilst it is constant
	\begin{quote}
	{\itshape That's a mistake on our part, apologies. We have corrected this in {\bf Appendix A}, where it now appears.}
\end{quote}
	\item As the building blocks are given expressions (eq 10 and 11 for instance) it would be nice to mention the fact that the elements in these will be characterize later (otherwise the reader thinks she/he should already know what the expression of, say, UsAA is)
\begin{quote}
	{\itshape As suggested by this reviewer and by the editor, we have now moved up the two stage example, so that we show the reader what these matrices can look like immediately after introducing them. Again shifting the focus from mathematical derivations to an explicit example, thereby hopefully improving the manuscripts' accessibility/readability.  }
\end{quote}

\end{itemize}

\noindent -- Line 19: I find it weird to write that the "process of adaptation can be impeded by ...changing environmental conditions".

\begin{itemize}
	\item One organism can be adapted to an environment (as in a series of environmental parameters) but also to changes in that environment (plasticity, genetic variance etc ...). It is actually from this consideration that a lot of studies on adaptation and adaptive dynamics.
	\item I think this sentence should be rewritten to make it clear that it is not changes in environmental conditions per se that impede the adaptive process but some of their characteristics (speed, range etc.)
\end{itemize}

\begin{quote}
	{\itshape Very good point, that was awkward wording on our part. We have rephrased to more clearly explain that it is primarily the speed/magnitude of environmental change that we are referring to ({\bf L.20}). We have also included a new citation of a recent AmNat paper that explores some of these issues explicitly.}
\end{quote}


\noindent -- Line 23: The link between genetic constraints and genetic trade-offs should be clarified. Or rather a definition of each term provided, for the reader to understand this sentence. I suspect that, for many a reader of AmNat, genetic constraints and genetic trade-offs are equivalent concepts.

\begin{quote}
	{\itshape Agreed. We have rephrased to avoid the term "trade-off" which tends to over-simplify the concept in or opinion ({\bf L.25}).}
\end{quote}

\noindent -- Line 30: "fitness trade-offs ... provide ...mechanism for the maintenance of genetic variation". This requires a few words of explanation. Is that a general statement about (fitness) trade-offs? or is it actually particular to SA trade-offs or to trade-offs between fixed heterogeneity components?
\begin{itemize}
	\item For instance, a fitness trade off between fertility at 3 years and fertility at 6 years will have no positive effect on genetic variance. To the contrary, it may lead the organism to evolve an optimal strategy accounting for overall fitness and the trade-offs, vis-à-vis investments in reproduction at those different stages in the life cycle. And as such this may lead to the selection of individuals with such a strategy and therefore to low genetic variance.
\end{itemize}

\begin{quote}
	{\itshape We have rephrased the first sentence of this paragraph to be more specific ({\bf L.27}), which we hope also addresses this comment.}
\end{quote}


\noindent -- Line 264: this is an important point that is a bit unclear here, according to me. Could you clarify that what you call demographic rates are the transitions in the overall model and so include both demography and genetics? Clarify also that the model is not density dependent because it only depends on frequencies ... and maybe relate it to a general equation where A does not depend on n per se but on the q frequencies.

\begin{quote}
	{\itshape \textcolor{red}{Lotte, double check here?} We have added a few lines to discuss frequency vs density dependence when we first introduce the population projection matrix in equation 3 (CHECK LINE NUMBERS AFTER WE"RE DONE EDITING ). In addition, we have tried to clarify to improve this section by discussing that the survival and fertility rates ("demographic rates") are density-indepedent, and that the allele frequency dynamics make the model frequency dependent.}
\end{quote}

- Line 35: SA selection arises also from non-genetic trade-offs, which makes the formulation of that sentence a bit weird.

\begin{quote}
	{\itshape We have rephrased this sentence to clarify and be more consistent with our language earlier in the preceding paragraphs ({\bf L.37--39}).}
\end{quote}

- Lines 289-292: inbreeding depression is expressed as a function of C. However, C is not written in the equation but L which depends on C. C is the introduced parameter and so should appear. If the expression is too complicated then it should be described literally, but there is no need to introduce L here, in any case.

\begin{quote}
	{\itshape We have removed $L$ and written the full expression involving $C$.}
\end{quote}

- Line 42: Define analogous here

\begin{quote}
	{\itshape This sentence has been rephrased.}
\end{quote}

- Line 54: maybe a few words to remind the reader about what dioecious and hermaphrodite organisms are.

\begin{quote}
	{\itshape We have added a sentence earlier in this paragraph where we remind readers what dioecy is ({\bf L.54}). Hermaphroditism is defined earlier in the paragraph, so an additional reference seemed redundant.}
\end{quote}

- Line 80: just a few words to clarify "obligate outcrossing" (something like (no selfing allowed) maybe)

\begin{quote}
	{\itshape Fixed ({\bf L.103-104}).}
\end{quote}

- Line 233: f and f’ are called "adult fertilities". However, they correspond to that rate only for certain specific genotypes. Maybe clarify that by calling them "adult maximum fertilities" ?

\begin{quote}
	{\itshape Good point, fixed ({\bf L.161, 185}).}
\end{quote}

- Results: it took me a while to figure out Fig 1 (mainly because I was looking at it in B and W). I am just wondering whether it would be easier to read with only one f per chart. Maybe as a separate chart that illustrates the general shape of the parameter space for polym * extinction for only one C, one h and one f... then followed by these 6 graphs as they are now.

\begin{quote}
	{\itshape We want to thank the reviewer for this suggestion. We now start with a figure that only has one f, and that separately explains what happens to the allele frequencies (Population genetic outcome) and to the population growth rate (Demographic Outcome), see new {\bf fig. 1}. We hope that this will make it easier to interpret the old fig. 1, which is now {\bf fig. 2}. We have also included a new supplementary figure, {\bf fig. S1}, with detailed heatmaps for each of the parameter combinations presented in the this figure (i.e., one $f$ per panel). However, we felt that this revised figure was to large and complicated to present in the main text (18 vs. 6 panels).}
\end{quote}

\noindent-- Maybe mention once, in a few words, what is meant by "segregate as a diallelic locus".
\begin{quote}
	{\itshape We have added a brief parenthesis to clarify ({\bf L.434--435}).}
\end{quote}

\noindent-- Fig 4 : they are supposed to be dashed lines and dotted lines but all I see are dashed lines with various dashes. I found 4B in particular, hard to understand at first sight.

\begin{quote}
	{\itshape This figure now appears as four separate heatmaps, each with one set of invasion conditions ({\bf fig.~5}). We hope this eases interpretation of the figure.}
\end{quote}
\bigskip

\noindent Typos:
\smallskip

\noindent-- The 0s representing 0 matrices in eq 5 are not bold
\begin{quote}
	{\itshape Fixed}
\end{quote}

\noindent-- In same equation 1w is not defined
\begin{quote}
	{\itshape Fixed}
\end{quote}

\noindent-- Line 142, 2 "the"s
\begin{quote}
	{\itshape Fixed}
\end{quote}

\noindent-- Line 553 : not a typo, but the "was that how" in the middle of the sentence makes it hard to read
\begin{quote}
	{\itshape We have re-written this sentence to avoid this awkward phrasing ({\bf L.579--580}).}
\end{quote}


\end{document}