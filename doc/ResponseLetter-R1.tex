%===========================================
% Preamble

\documentclass[11pt]{article}


%**********
%Dependencies

%**********
%Dependencies
%\usepackage[left]{lineno}
\usepackage{titlesec}
\usepackage{amsmath}
\usepackage{amsfonts}
\usepackage{amssymb}
%\usepackage[utf8]{inputenc}
\usepackage{color,soul}
\usepackage[sc]{mathpazo} %Like Palatino with extensive math support
\usepackage{fullpage}
\usepackage[authoryear,sectionbib,sort]{natbib}
\linespread{1}
\usepackage[utf8]{inputenc}
\usepackage{lineno}
\usepackage[hidelinks]{hyperref}


% New commands: fonts
\newcommand{\code}{\fontfamily{pcr}\selectfont}
\newcommand*\chem[1]{\ensuremath{\mathrm{#1}}}

\renewcommand\textbullet{\ensuremath{\bullet}}
%===========================================
% Title page


% \title{Cover Letter}
% \author{Colin Olito}
% \date{\today}


\begin{document}
%\maketitle
%\newpage{}


%===========================================
% Document


\section*{}
\noindent To the AE, Ed., \& Editorial Board
\bigskip

\noindent Please consider this revision of our manuscript (MS \#60734), titled "The demographic costs of sexually antagonistic selection in partially selfing populations" for publication in \textit{The American Naturalist}. 
\bigskip

\noindent First of all, we want to thank the AE, Ed., and EiC for their thoughtful and forthright responses (and apologies) regarding the editorial delay on our manuscript. We understand completely that pandemic-related personal issues have been and continue to be extremely disruptive in these difficult times, and that editorial duties for our journals is a volunteer service. To our AE, Dr.~Orive, we sincerely hope that you and your family are doing ok, and we truly appreciate the time and effort you took to evaluate our paper while you clearly had more imporant issues to deal with.
\bigskip

\noindent We received very helpful and constructive comments on our first submission from Associate Editor Dr.~Maria Orive, Editor Erol Ak\c{c}ay, and two reviewers, and have now made a variety of major revisions to the MS in order to address their concerns. In particular, we have ... (1) made significant changes to our introduction and discussion in an attempt to engage with a broader readership by developing conceptual links between ours and previous work on evolutionarily explicit structured models and sexually antagonistic selection; (2) attempted to lessen the burden on non-theoretical readers by moving most of the development of the general 2-stage model to the Supplementary Material and following R2's many helpful suggestions regarding the presentation of the model; and (3) tried to more fully address the impacts of density dependence in three ways:  ... \textcolor{blue}{({\itshape i}) by revising our figures to illustrate predicted absolute population intrinsic growth rates rather than simply the boundary between demographically viable \& inviable regions} ; ({\itshape ii}) by including a brief supplement with a direct comparison between our current density-independent predictions and a new version of our model which includes simple DD; and ({\itshape iii}) by expanding our discussion about the implicaitons of DD to address these additions. Overall, our changes have shortened the manuscript by \textcolor{blue}{some impressive amount}, but we feel that they have greatly improved the paper. 
\bigskip

\noindent We have provided detailed responses to each of the editors' and reviewers' comments below. For clarity, we use bold face for all line and figure references to the revised ms in our responses. We have also used blue text in the revised manuscript file to help identify the key additions we have made to the revised text.
\bigskip

\noindent Thank you again for efforts and consideration.
\bigskip

\noindent Sincerely,

\noindent The Authors \\
%\noindent Colin Olito \& Charlotte DeVries \\
\noindent\makebox[\linewidth]{\rule{\textwidth}{0.4pt}}
\smallskip


%%%%%%%%%%%%%%%%%%%%
\section*{Decision letter with our responses}

\subsection*{Responses to Editor Erol Ak\c{c}ay}


I must first apologize for the time it took to return this decision, and thank you for your patience. Even in the best of times, unexpected events and circumstances cause delays for reviewers and editors, and many of us are still operating under the continuing strain of the pandemic. In this case, I must also apologize to have added to the delay personally, which is because of a backlog that built up due to disruptions that happened in the last couple of months that ordinarily might not have led to huge issues, but after a year and a half of the pandemic seem to snowball out of control.

\begin{quote}
	{\itshape Thank you, again, for your candid and forthright apology and explanation. We truly appreciate it, and sympathize with the growing challenges that seem to creep up on us all during these difficult times.}
\end{quote}


The two reviewers, Dr. Orive, and myself all read your paper with great interest. We are all agreed that your model is well-constructed, and well-explained, and your conclusions are clearly presented. The insight that sexually antagonistic genetic variation might depress population demographic rates and affect population viability is a valuable one. We also appreciate that you use empirical data from a case study to parameterize your model, and agree that this is very useful to judge the real world relevance of your calculations. As such, there is good potential for this manuscript to be a great American Naturalist paper.

\begin{quote}
	{\itshape This is very encouraging indeed, and we certainly hope that our revisions can successfully fulfill the potential you see in our paper.}
\end{quote}

As you will see in Dr. Orive’s excellent summary of the reviews, one issue that we feel you need to address is to make the paper more suitable for the general audience of the American Naturalist. To my mind, this has two parts: First, I think you can reduce the amount of space you devote to the general model in the main text (this can go into the SI as a generalization and also as a reference for the later three-stage model for Mimulus), as most of the biological conclusions actually come from the two-stage model which requires less notation and the construction of which is more easily explained. Although I personally appreciated the construction of the general model, I think for our non-theoretical readership, the cost/benefit ratio of reading through the general model (which is not used for the numerical calculations) is too great. 

\begin{quote}
	{\itshape ...}
\end{quote}

Second, like Reviewer 1 and Dr. Orive, I do think you should better connect your model to the broader topics of evolution in structured populations and sexual antagonism. That would give your paper a broader impact in terms of readership.

\begin{quote}
	{\itshape We agree entirely. Please see our detailed responses to Dr.~Orive and R1 on this count.}
\end{quote}


For my own part, I do have one additional major comment, which concerns how you decide on demographic viability. As you are admirably explicit in the model construction, this is a linear projection model with no density dependence in demographic rates. As such, the population growth or decline you infer from a given parameter set is necessarily "local," i.e., it applies around a particular population density where the base demographic rates are measured or modeled. To my mind, this is a potentially major gap between model calculations and what one can say about viability of actual populations in the wild, which will have density dependence and therefore will face varying demographic rates with density. For example you can have fertility rates that give a growth rate less than 1 at some densities, but as density becomes low, mortality might decrease (or fertility increase) to bring growth rates back to one. (That is of course the assumption of most eco-evolutionary theory such as adaptive dynamics: some sort of density dependence is assumed to keep populations at demographic equilibrium while new traits invade.) I understand that incorporating such non-linear, density dependent feedbacks would make the model complicated, but I also think that their lack in the current model does dampen your conclusions, in the sense that they don’t necessarily mean that populations would not be able to persist if their growth rate at some undefined density becomes less than one. At the very least, this is an issue that deserves acknowledgment and fairly extensive discussion because it directly affects your major conclusion. I was disappointed that the issue only came up once in passing in the discussion. Ideally, you could present some numerical simulations with density dependence. I sorta suspect that what you might get (at least for some cases) is depressed population densities rather than outright population inviability. I don’t know how difficult these simulations would be, but at least some numerical exploration might not be too much out of the way.

\begin{quote}
	{\itshape This is a very good and totally valid point. To be honest, we believe that a separate paper is probably necessary to deal adequately with the many consequences of density dependence for our model predictions. However, we also agree that sidestepping this issue completely isn't good enough. In the end, we have made three main changes to address density dependence. First, we have taken your advice in the following comment and revised our figures 1 \& 4 to show absolute intrinsic growth rates across selection parameter space rather than just the boundary between demographically viable \& inviable parameter space (\textcolor{blue}{caveat about Fig.1 vs. 4?}). Second, we have included a new supplementary appendix to the paper where we briefly explain how density dependence can be incorporated into the model, and provide a direct comparison between our current density-independent predictions and a new version of our model which includes DD logistic growth. Finally, we have expanded our results and discussion to more fully address the implications of DD for our model, with reference to our modified figures and the new supplement.}
\end{quote}


Related to that, and without doing more modeling, one quantity of interest you can present is the actual growth rate, and not just the region of viability/inviability. In a world where we acknowledge density dependence as a possibility, how far above or below one growth rate is can be informative about whether you actually expect extinction or depressed densities.

\begin{quote}
	{\itshape This is great suggestion. As mentioned above, we have changed figures 1 \& 4 to include a heatmap of absolute values for predicted $\lambda$ instead of the extinction boundaries. \textcolor{blue}{However, one unfortunate consequence of presenting $\lambda$ instead of the extinction boundaries is that it is no longer possible to illustrate multiple values of maximum fertility ($f$) on each figure panel. In particular, Figure 1, now requires 18 panels illustrating $\lambda$ for the same parameter combinations. We felt that this resulted in an unnecessarily complicated figure, especially since Figure 1's main purpose was to briefly illustrate the general behaviour of the model. For simplicity, we have decided to include this modified Figure 1 as a supplementary figure. However, we now present the modified version of Figure 4 in the main text}.}
\end{quote}


In addition to these general points, both reviewers and Dr. Orive have numerous helpful suggestions and corrections that will no doubt be useful while you revise the manuscript, and I urge you to pay close attention to them.
\bigskip


\noindent xxxxxxxxxxxxxxxxxxxxxxxxxxxxxxxxxxxxxxxxxxxxxxxxx

\section*{Responses to Associate Editor Maria Orive}

Both reviewers found the work interesting and potentially to be of strong interest to the readership of AmNat, and I agree. However, Reviewer 1 brings up an important point regarding the framing of the work and placing it within a broader context of the literature on evolutionarily explicit structured models, which I think is important for the authors to do in order to make the paper accessible to and of interest for the overall readership of AmNat. There are other points that the reviewers bring up that are important to consider, but I think this one needs to be addressed carefully.

\begin{quote}
	{\itshape ...}
\end{quote}


Reviewer 2 has some very nice suggestions regarding ideas for simplifying and clarifying the model description.

\begin{quote}
	{\itshape ...}
\end{quote}

In addition to the clarifications/edits and other points raised by the reviewers, I include a handful of minor edits/typos here of my own:

\noindent (1) Abstract, perhaps add a very general and short explanation of the term sexually antagonistic polymorphism.
\begin{quote}
	{\itshape ...}
\end{quote}

\noindent (2) Line 165; "fertility"
\begin{quote}
	{\itshape Fixed.}
\end{quote}

\noindent (3) Table 1, definition of inbreeding depression terms; it would be clearer to define each of the 4 terms separately, "Inbreeding depression terms for ovule viability, juvenile survival, adult survival, and transition rates respectively"
\begin{quote}
	{\itshape Fixed}
\end{quote}

\noindent (4) \textcolor{red}{Figure 1; I generally prefer axes labels to have clear meanings, rather than just symbols – here, the vertical and horizontal axes give the female (vertical) vs. male (horizontal) selection on reproductive function.}
\begin{quote}
	{\itshape ...}
\end{quote}

\noindent (5) Lines 305, 306, and 478; the use of the term "fixates" rather than "fixes" is not standard, I would change to "fixes" in all cases.
\begin{quote}
	{\itshape Fixed \textcolor{red}{Note to Colin: I find that fixes sounds a bit odd? I changed it to "goes to fixation" in one instance. I trust your final judgement on this as our native.}}
\end{quote}

\noindent (6) \textcolor{red}{Figure 4; same comment as for Figure 1.}
\begin{quote}
	{\itshape ...}
\end{quote}

Finally, I thank the authors for their patience; the delay in receiving these reviews lies not with the reviewers, but with myself, due to some personal and family issues that have recently arisen.

\begin{quote}
	{\itshape Thank you again for your efforts and consideration in spite of these challenges. We really appreciate it.}
\end{quote}


\noindent xxxxxxxxxxxxxxxxxxxxxxxxxxxxxxxxxxxxxxxxxxxxxxxxx

\subsection*{Reviewer \#1:}

The manuscript reports the derivation, analysis, and example application of an extended frequency-dependent matrix model to explore links between sexual antagonistic selection, population persistence, and the maintenance of heritable variation. The paper is very well written and is straightforward to follow. The derivations are correct, although I did not try to derive all the equations in the appendices. My one comment is how might environmental stochasticity in the rate of selfing and outcrossing impact general conclusions? Other than that, I have no concerns about the science that is reported. My concerns rate the very narrow focus of the paper as written.

The manuscript adds to a growing and important literature where eco-evolutionary feedbacks are examined in structured models, some of which incorporate evolution explicitly. It also contributes nicely to a body of empirical and theoretical work on sexually antagonistic selection.

\begin{quote}
	{\itshape Thank you for the encouraging words.}
\end{quote}

As written the manuscript will not appeal to the broad readership of the American Naturalist because the weakness of the manuscript is the author makes little attempt to fit the work into the broader relevant literature. Instead, the paper has been written for a small audience of experts interested in the demographic consequences of sexually antagonistic selection. As written, the work is suited to a journal such as Journal of Theoretical Biology. This is disappointing, and the authors do themselves no favors by writing for such a niche audience. The manuscript could be well-suited to the American Naturalist if the authors were to engage with the broader literature. This can be done by extending the introduction to link this work to the wider literature on evolutionarily explicit structured models, and to the large empirical literature on sexually antagonistic selection. If they are able to do this appropriately, I believe this paper would make a strong contribution to the American Naturalist.

\begin{quote}
	{\itshape This is a really important comment and we thank you pointing out this major shortcoming. One of our main goals while crafting the first draft of the manuscript was to find a middle ground that would appeal to both demographers and people studying sexual conflict. It seems clear now that we didn't quite achieve that goal. Rather, by making compromises in the introduction we ended up lacking scholarship on both sides. To address your main concern, we have restructured our Introduction to more directly connect with the broader literature on evolutionarily explicit structured models (see {\bf L.XXX--XXX}) and now try to link our overview of sexually antagonistic selection theory with the existing empirical literature (see {\bf L.XXX--XXX}). We hope these changes succeed in broadening the appeal of our manuscript to the wider readership of AmNat.}
\end{quote}

\noindent xxxxxxxxxxxxxxxxxxxxxxxxxxxxxxxxxxxxxxxxxxxxxxxxx


\subsection*{Reviewer \#2:}

I have read, with great interest, this manuscript analyzing the effects of sexually antagonistic selection in hermaphrodites at both the population genetics and population dynamics levels. As a representative (I guess) of the general readership of AmNat and also as a population dynamics specialist, I appreciated the description of the model, its parametrization, the way it is constructed and analyzed. Not being a population genetics specialist, it is harder for me to comment on that side of the interpretations, apart from generic comments and in the light of the model’s characteristics. Overall, my take is that this paper is interesting, well written, the model well built and the ideas quite novel; and that it therefore corresponds to what readers except to find in AmNat both in terms of content and quality.

I have one main comment:
-- Discussion: Conclusions are drawn before limitations of the model are mentioned. However, it seems to me that it would be very important to discuss the effect of those limitations on the results themselves and therefore discuss the generality of the results. For instance, what is the potential effect of female dominance in the marriage function on the results and in particular on conclusion of lines 477-480

\begin{quote}
	{\itshape ...}
\end{quote}

\noindent Apart from that, I have only minor comments, that I list below (roughly in decreasing order of importance):
\bigskip 

\noindent -- Model: lines 114-195
\begin{itemize}
	\item The matrices H are mentioned before they are introduced. (Therefore, the parenthesis in lines 123-124 cannot be understood by the reader). And then are only defined in the following section	
	\begin{itemize}
		\item It would make sense to reorder the model presentation by presenting the building blocks (arguably U, F and H matrices, but I'll come back to H in the next point), what they do, what they are for; then provide some properties (F independent from selfing/crossing, H from stage, etc ...) and then go into the details of their construction
	\end{itemize}
	\item The main issue I have here, is that H is introduced, but then disappears from the output (see equations 12 and 13 for instance that use H but it does not appear, or the general output eq 9)
	\begin{itemize}
		\item This suggests combining F and H directly into an object.
	\end{itemize}
\end{itemize}

\begin{itemize}
	\item More generally, the matrix formulations of the model may be a bit hard to read for the general reader of AmNat. Getting rid of H (or to the contrary, using it all the way, thus simplifying eq 12 and 13 for instance) and not giving a name to matrix W (which is just a bunch of 0s 1s and 1/2s) are steps that can make it easier for the reader to follow.
\begin{quote}
	{\itshape Response to the above 5 points: As per the reviewers (and the editors) suggestion, we have moved a lot of the derivations into the supplementary materials, and now no longer use H or W in the main text. In doing so, we hope to shift the focus from mathematical derivations, which are probably unhelpful to many readers, towards explaining the end result instead. }
\end{quote}

	\item It is quite confusing that HX is written as a constant (eq 7) whilst it depends on the frequencies and that HS is written as dependent (eq6) whilst it is constant
	\begin{quote}
	{\itshape That's a mistake on our part, apologies.}
\end{quote}
	\item As the building blocks are given expressions (eq 10 and 11 for instance) it would be nice to mention the fact that the elements in these will be characterize later (otherwise the reader thinks she/he should already know what the expression of, say, UsAA is)
\end{itemize}

\begin{quote}
	{\itshape As suggested by this reviewer and by the editor, we have now moved up the two stage example, so that we show the reader what these matrices can look like immediately after introducing them. Again shifting the focus from mathematical derivations to an explicit example, thereby hopefully improving the manuscripts' accessibility/readability.  }
\end{quote}
\noindent -- Line 19: I find it weird to write that the "process of adaptation can be impeded by ...changing environmental conditions".

\begin{itemize}
	\item One organism can be adapted to an environment (as in a series of environmental parameters) but also to changes in that environment (plasticity, genetic variance etc ...). It is actually from this consideration that a lot of studies on adaptation and adaptive dynamics.
	\item I think this sentence should be rewritten to make it clear that it is not changes in environmental conditions per se that impede the adaptive process but some of their characteristics (speed, range etc.)
\end{itemize}

\begin{quote}
	{\itshape ... our response}
\end{quote}


\noindent -- Line 23: The link between genetic constraints and genetic trade-offs should be clarified. Or rather a definition of each term provided, for the reader to understand this sentence. I suspect that, for many a reader of AmNat, genetic constraints and genetic trade-offs are equivalent concepts.

\begin{quote}
	{\itshape ... our response}
\end{quote}

\noindent -- Line 30: "fitness trade-offs ... provide ...mechanism for the maintenance of genetic variation". This requires a few words of explanation. Is that a general statement about (fitness) trade-offs? or is it actually particular to SA trade-offs or to trade-offs between fixed heterogeneity components?
\begin{itemize}
	\item For instance, a fitness trade off between fertility at 3 years and fertility at 6 years will have no positive effect on genetic variance. To the contrary, it may lead the organism to evolve an optimal strategy accounting for overall fitness and the trade-offs, vis-à-vis investments in reproduction at those different stages in the life cycle. And as such this may lead to the selection of individuals with such a strategy and therefore to low genetic variance.
\end{itemize}

\begin{quote}
	{\itshape ... our response}
\end{quote}


\noindent -- Line 264: this is an important point that is a bit unclear here, according to me. Could you clarify that what you call demographic rates are the transitions in the overall model and so include both demography and genetics? Clarify also that the model is not density dependent because it only depends on frequencies ... and maybe relate it to a general equation where A does not depend on n per se but on the q frequencies.

\begin{quote}
	{\itshape ... our response \textcolor{red}{I cannot figure out where in the manuscript this is, I don't think I have the exact R0 version as it was submitted?}}
\end{quote}

- Line 35: SA selection arises also from non-genetic trade-offs, which makes the formulation of that sentence a bit weird.

\begin{quote}
	{\itshape ... our response}
\end{quote}

- Lines 289-292: inbreeding depression is expressed as a function of C. However, C is not written in the equation but L which depends on C. C is the introduced parameter and so should appear. If the expression is too complicated then it should be described literally, but there is no need to introduce L here, in any case.

\begin{quote}
	{\itshape ... our response}
\end{quote}

- Line 42: Define analogous here

\begin{quote}
	{\itshape ... our response}
\end{quote}

- Line 54: maybe a few words to remind the reader about what dioecious and hermaphrodite organisms are.

\begin{quote}
	{\itshape ... our response}
\end{quote}

- Line 80: just a few words to clarify "obligate outcrossing" (something like (no selfing allowed) maybe)

\begin{quote}
	{\itshape Fixed}
\end{quote}

- Line 233: f and f’ are called "adult fertilities". However, they correspond to that rate only for certain specific genotypes. Maybe clarify that by calling them "adult maximum fertilities" ?

\begin{quote}
	{\itshape Good point, fixed.}
\end{quote}

- Results: it took me a while to figure out Fig 1 (mainly because I was looking at it in B and W). I am just wondering whether it would be easier to read with only one f per chart. Maybe as a separate chart that illustrates the general shape of the parameter space for polym * extinction for only one C, one h and one f... then followed by these 6 graphs as they are now.

\begin{quote}
	{\itshape ... our response}
\end{quote}

\noindent-- Maybe mention once, in a few words, what is meant by "segregate as a diallelic locus".
\begin{quote}
	{\itshape ... our response}
\end{quote}

\noindent-- Fig 4 : they are supposed to be dashed lines and dotted lines but all I see are dashed lines with various dashes. I found 4B in particular, hard to understand at first sight.

\begin{quote}
	{\itshape ... our response}
\end{quote}
\bigskip

\noindent Typos:
\smallskip

\noindent-- The 0s representing 0 matrices in eq 5 are not bold
\begin{quote}
	{\itshape Fixed}
\end{quote}

\noindent-- In same equation 1w is not defined
\begin{quote}
	{\itshape Fixed}
\end{quote}

\noindent-- Line 142, 2 "the"s
\begin{quote}
	{\itshape Fixed}
\end{quote}

\noindent-- Line 553 : not a typo, but the "was that how" in the middle of the sentence makes it hard to read
\begin{quote}
	{\itshape ... our response}
\end{quote}


\end{document}