\documentclass[11pt]{article}
\usepackage[sc]{mathpazo} %Like Palatino with extensive math support
\usepackage{fullpage}
\usepackage[authoryear,sectionbib,sort]{natbib}

\bibliographystyle{amnatnat.bst}
\setlength{\bibsep}{0.0pt}
\linespread{1.5}
\usepackage[utf8]{inputenc}
\usepackage{lineno}
\usepackage{titlesec}
\usepackage{amsmath}
\usepackage{amsfonts}
\usepackage{amssymb}
\titleformat{\section}[block]{\Large\bfseries\filcenter}{\thesection}{1em}{}
\titleformat{\subsection}[block]{\Large\itshape\filcenter}{\thesubsection}{1em}{}
\titleformat{\subsubsection}[block]{\large\itshape}{\thesubsubsection}{1em}{}
\titleformat{\paragraph}[runin]{\itshape}{\theparagraph}{1em}{}[. ]
\usepackage{fancyhdr}
\usepackage{color,soul}
\pagestyle{fancy}
\usepackage[colorlinks=true, allcolors=black]{hyperref}
\setlength{\headheight}{13.6pt}
\renewcommand{\refname}{References}

\usepackage{tikz}

% \newcommand\encircle[1]{%
%   \tikz[baseline=(X.base)] 
%     \node (X) [draw, shape=circle, inner sep=0] {\strut #1};}
\newcommand*\circled[1]{\tikz[baseline=(char.base)]{
            \node[shape=circle,draw,inner sep=2pt] (char) {#1};}}

% Equation numbering
\newcommand\numberthis{\addtocounter{equation}{1}\tag{\theequation}}

% Graphics package
\usepackage{graphicx}
\graphicspath{{../output/figs/}.pdf}

% Change default margins
\usepackage[top=0.75in, bottom=0.75in, left=0.75in, right=0.75in]{geometry}

% Definitions
\def\mathbi#1{\textbf{\em #1}}
\def\mbf#1{\mathbf{#1}}
\def\mbb#1{\mathbb{#1}}
\def\mcal#1{\mathcal{#1}}
\newcommand{\bo}[1]{{\bf #1}}
\newcommand{\tr}{{\mbox{\tiny \sf T}}}
\newcommand{\bm}[1]{\mbox{\boldmath $#1$}}
\newcommand{\om}{\omega}
\DeclareRobustCommand{\firstsecond}[2]{#2}
 \def\linenumberfont{\normalfont\scriptsize}
 \newcommand{\kron}{\otimes}
\usepackage{pdflscape}

%%%%%%%%%%%%%%%%%%%%%
% Header
%%%%%%%%%%%%%%%%%%%%%
%
% Customize the line below with the last name of your first author and
% the short title of your MS. You can comment authorship information out
% while your MS is undergoing double-blind review.
%
\rhead{Supplement to Olito \& de Vries, "Costs of SA selection" \textit{Am.~Nat.}}
\setlength{\headsep}{0.3in}  
\lhead{} 

%%%%%%%%%%%%%%%%%%%%%
% Line numbering
%%%%%%%%%%%%%%%%%%%%%
%
% Please use line numbering with your initial submission and
% subsequent revisions. After acceptance, please turn line numbering
% off by adding percent signs to the lines %\usepackage{lineno} and
% to %\linenumbers{} and %\modulolinenumbers[3] below.
%
% To avoid line numbering being thrown off around math environments,
% the math environments have to be wrapped using
% \begin{linenomath*} and \end{linenomath*}
%
% (Thanks to Vlastimil Krivan for pointing this out to us!)

\title{Online Supplement to "Demographic costs of sexually antagonistic selection in partially selfing populations \\ 
%\LaTeX{} Template for Author-Supplied Supplementary PDFs, \\ 
\textit{The~American~Naturalist} }

% This version of the LaTeX supplementary template was last updated on
% November 8, 2019.

%%%%%%%%%%%%%%%%%%%%%
% Authorship
%%%%%%%%%%%%%%%%%%%%%
% Please remove authorship information while your paper is under review,
% unless you wish to waive your anonymity under double-blind review. You
% will need to add this information back in to your final files after
% acceptance.
%
% Once accepted for publication, author-supplied PDFs should have a 
% title page that includes (at least) the authors' names, the title of 
% the MS, and the name of the journal. It should also have a header and
% page numbers.
\author{Colin Olito$^{1,\ast}$ \\ 
Charlotte de Vries$^{2}$}

\date{\today}

\begin{document}

\maketitle

\noindent{} 1. Department of Biology, Lund University, Lund 223 62, Sweden;

\noindent{} 2.  Department of Evolutionary Biology and Environmental Studies, University of Zurich, Zurich CH-8057, Switzerland;

\noindent{} $\ast$ Corresponding author; e-mail: \url{colin.olito@gmail.com}

%\linenumbers{}
%\modulolinenumbers[3]


% In many cases, The American Naturalist allows authors to typeset their
% own supplementary material in an author-supplied PDF. This template
% applies to such cases. 
% 
% For appendices that will be typeset by the AmNat editorial staff, 
% please see the main LaTeX template, available from
% https://www.journals.uchicago.edu/journals/an/instruct
% Such appendices typically include descriptions of methods and tables
% defining parameters.
%
% In general, you have wide discretion for how you want to format an
% author-supplied PDF. They should in any case have a title page, 
% page numbers, and a header identifying the MS's (short) title.
%
% Counters for the online supplement should normally begin with an S
% (thus normally figure S1, figure S2, table S1, equation S1, etc.).

\renewcommand{\theequation}{S\arabic{equation}}
% redefine the command that creates the equation number.
\renewcommand{\thetable}{S\arabic{table}}
\renewcommand{\thesection}{S\arabic{section}}
\renewcommand{\thefigure}{S\arabic{figure}}
\setcounter{equation}{0}  % reset counter 
\setcounter{figure}{0}
\setcounter{table}{0}

% In online supplementary PDFs, sections can be numbered or not
% (at your discretion). If they are numbered, sections should usually
% begin with an S.

\newpage{}
\section{Construction of the population projection matrix}
We will now derive the expressions for the survival and fertility matrices $\mcal{U}^S$, $\mcal{U}^X$, $\mcal{F}^S$, and $\mcal{F}^X$ as given by equations \textcolor{red}{(10)-(13)} in the main text. The matrices $\mcal{U}^S$, $\mcal{U}^X$, $\mcal{F}^S$, and $\mcal{F}^X$ are the ingredients of the population projection matrix, $\tilde{\mbf{A}}$, as given in equation \textcolor{red}{(9)}, repeated here for convenience, 
\begin{linenomath*}
\begin{equation} \label{eq:Atilde_appendix}
	\tilde{\mbf{A}}[\tilde{\mbf{n}}] = 
			\underbrace{\left(
			\begin{array}{c|c}
				\mcal{U}^S & \mbf{0} \\ \hline
				\mbf{0} & \mcal{U}^X \\
			\end{array} \right)}_{\tilde{\mbf{U}}} + 
			\underbrace{\left(
			\begin{array}{c|c}
				\mcal{F}^S C(1 - \delta) & \mcal{F}^S(\tilde{\mbf{p}}) C(1 - \delta) \\ \hline
				\mcal{F}^X(\tilde{\mbf{p}}) (1 - C) & \mcal{F}^X(\tilde{\mbf{p}}) (1 - C)\\
			\end{array} \right)}_{\tilde{\mbf{F}}}.
\end{equation}
\end{linenomath*}

Our derivation follows closely the derivations in \cite{deVriesCaswell2019a} and \cite{deVriesCaswell2019b}, but extended to model mixed mating systems. To construct $\tilde{\mbf{A}}[\tilde{\mbf{n}}]$ using the vec-permutation approach \citep{CaswellEtAl2018}, we create a set of block-diagonal matrices $\mathbb{U}^S$, $\mathbb{U}^X$, $\mathbb{F}^X$, $\mathbb{F}^S$, and $\mathbb{H}^S$, $\mathbb{H}^X$, by putting the corresponding matrices on the diagonal. These block diagonal matrices can be written as
\begin{linenomath*}
\begin{eqnarray}
 \mathbb{U}^S&=&\sum_{i=1}^g \mathbf{E}_{ii} \otimes \mathbf{U}^S_{i}, \label{eq:US_app}\\
 \mathbb{F}^S&=&\sum_{i=1}^g \mathbf{E}_{ii} \otimes \mathbf{F}^S_{i}, \label{eq:FS_app}\\
  \mathbb{H}^S &=& \bo I_\om \kron \bo H^S , \label{eq:HS_app} \\
      \mathbb{D} &=& \bo I_\om \kron \bo I_g, \label{eq:D_app} 
\end{eqnarray}
\end{linenomath*}
where $\mathbf{E}_{ii}$ is of dimensions $g \times g$. Analogous equations can be written down for $ \mathbb{U}^X$, $ \mathbb{F}^X$, and $ \mathbb{H}^X$.  
% \begin{linenomath*}
% \begin{eqnarray} 
% 	\mbb{U}^S &= &
% 		\left(
% 			\begin{array}{ccc}
% 				\mbf{U}^{S}_{AA} & \mbf{0} & \mbf{0} \\
% 				\mbf{0} & \mbf{U}^{S}_{Aa} & \mbf{0} \\
% 				\mbf{0} & \mbf{0} & \mbf{U}^{S}_{aa} \\
% 			\end{array} \right), \label{eq:BlkUS_appendix}\\
%   \mbb{F}^S &= &
% 		\left(
% 			\begin{array}{ccc}
% 				\mbf{F}^{S}_{AA} & \mbf{0} & \mbf{0} \\
% 				\mbf{0} & \mbf{F}^{S}_{Aa} & \mbf{0} \\
% 				\mbf{0} & \mbf{0} & \mbf{F}^{S}_{aa} \\
% 			\end{array} \right), \label{eq:BlkFS_appendix}\\
% \end{eqnarray}
% \end{linenomath*}
% Similarly construct block diagonal matrices from the parent-offspring genotype maps,
% \begin{linenomath*}
% \begin{eqnarray}
% 	\mathbb{H}^X(\tilde{\bo p}) = \bo I_\om \kron \bo H^X(\tilde{\bo p}),\\
% 	\mathbb{H}^S(\tilde{\bo p}) = \bo I_\om \kron \bo H^S(\tilde{\bo p}).
% \end{eqnarray}
% \end{linenomath*}

As in other multistate models, the survival and fertility matrices are then constructed from the following four block matrices, 
\begin{linenomath*}
\begin{eqnarray}
\mcal{U}^S &=&\bo K^\tr \mathbb{D} \bo K \mathbb{U}^S, \label{eq:KDKU_app}\\
\mcal{F}^S &=&\bo K^\tr \mbb{H}^S\bo K \mathbb{F}^S, \label{eq:KHKF_app}
\end{eqnarray}
\end{linenomath*}
where $\mathbf{K}=\mathbf{K}_{\omega,g}$ is the vec-permutation matrix \citep{HendersonSearle1981}. The vec-permutation matrix rearrangers the population vectors $\mathbf{n}^S$ and $\mathbf{n}^X$ so that $\mathbf{n}^S$ is ordered by genotype first, and then by stage, but $\mathbf{K} \mathbf{n}^S$ is organized by stage first, and then by genotype (see \cite{hunter2005use} or \cite{caswell2012matrix} for more on the vec-permutation method). 

The matrix $\mathbb{U}^S$ in equation (\ref{eq:KDKU_app}) generates transitions and survival of extant individuals within genotypes,  and acts on the the population vector of individuals produced through selfing, $\mathbf{n}^S$.. The matrix $\bo K$ then permutes the resulting vector so that it is now organised by stages first, and then by genotypes within stages. The block diagonal matrix $ \mathbb{D} $ then accounts for changes in genotype within stage classes of extant individuals. Since extant individuals generally do not change their genotype $\mathbb{D}$ will be an identity matrix in our model. Finally, the matrix $\bo K^\tr $ permutes the population vector once again so that it returns to the original organisation by genotype first and then by stage classes.  

Equation (\ref{eq:KHKF_app}) describes reproduction and genotype assignment of newborn individuals. The block matrix $\mathbb{F}^S$ contains fecundity rates for individuals of each genotype and stage combination and acts on the the population vector of individuals produced through selfing, $\mathbf{n}^S$. Initially, these offspring are associated with the genotype of their parent. The vec-permutation matrix $\bo K$ then rearranges the resulting vector so that it is organised by stage first and then by genotype. The block-diagonal matrix $\mbb{H}^S$ then allocates the offspring to their genotypes, according to equation (\ref{eq:HS}). Finally, $\bo K^\tr $  returns the vector to its initial orientation. 

Substituting equations (\ref{eq:US_app}) and (\ref{eq:D_app}) into equation (\ref{eq:KDKU_app}) yields
\begin{linenomath*}
\begin{eqnarray}
\mcal{U}^S &=& \sum_{i=1}^g \bo K^\tr \bo I_{\omega g}	 \bo K \mathbf{E}_{ii} \otimes \mathbf{U}^S_{i}, \\
 &=& \sum_{i=1}^g \mathbf{E}_{ii} \otimes \mathbf{U}^S_{i},
\end{eqnarray}
\end{linenomath*}
since $\bo K^\tr \bo I_{\omega g}	 \bo K=\bo I_{\omega g}$, or in block matrix form this yields equation (\ref{eq:BlkUS}), here repeated for convenience, 
\begin{linenomath*}
\begin{eqnarray}
\mcal{U}^S  = 
		\left(
			\begin{array}{ccc}
				\mbf{U}^{S}_{AA} & \mbf{0} & \mbf{0} \\
				\mbf{0} & \mbf{U}^{S}_{Aa} & \mbf{0} \\
				\mbf{0} & \mbf{0} & \mbf{U}^{S}_{aa} \\
			\end{array} \right).\label{eq:BlkUS}\\
\end{eqnarray}
\end{linenomath*}

Similarly, substituting equations (\ref{eq:FS_app}) and (\ref{eq:HS_app}) into equation (\ref{eq:KHKF_app}) yields,
\begin{linenomath*}
\begin{eqnarray}
\mcal{F}^S &=&\bo K^\tr \mbb{H}^S\bo K \mathbb{F}^S, \label{eq:KHKF_app}\\
 &=& \sum_{i=1}^g \bo K^\tr \left( \bo I_\om \kron \bo H^S \right)  \bo K \left( \mathbf{E}_{ii} \otimes \mathbf{F}^S_{i} \right).
\end{eqnarray}
\end{linenomath*}
Using $\bo K^\tr \left(\bo A \otimes \bo B  \right) \; \bo K=\bo B \otimes \bo A$ \citep{magnus1979commutation}, we obtain
\begin{linenomath*}
\begin{eqnarray}
\mcal{F}^S  &=& \sum_{i=1}^g  \left(  \bo H^S  \kron \bo I_\om\right) \left( \mathbf{E}_{ii} \otimes \mathbf{F}^S_{i} \right).
\end{eqnarray}
\end{linenomath*}
Finally note that $(\mathbf{A} \otimes \mathbf{B}) (\mathbf{C} \otimes \mathbf{D}) = \mathbf{A} \mathbf{C}\otimes \mathbf{B} \mathbf{D}$  to write
\begin{linenomath*}
\begin{eqnarray}
\mcal{F}^S  &=& \sum_{i=1}^g  \left(  \bo H^S \mathbf{E}_{ii} \right) \kron  \mathbf{F}^S_{i} .
\end{eqnarray}
\end{linenomath*}
Using the exact same sequence of equations, we can derive
\begin{linenomath*}
\begin{eqnarray}
\mcal{F}^X  &=& \sum_{i=1}^g  \left(  \bo H^X(\tilde{\mbf{p}}) \mathbf{E}_{ii} \right) \kron  \mathbf{F}^X_{i} .
\end{eqnarray}
\end{linenomath*}

Finally, substitute the expressions derived for the parent-offpspring genotype maps, $\bo H^X(\tilde{\mbf{p}}) $, and $\bo H^S$, in the main text (\textcolor{red}{equations (12) and (13))} into the above equations to obtain 
\begin{linenomath*}
\begin{eqnarray}
	\mcal{F}^S = \mbf{K}^{\intercal} \mbb{H}^S\mbf{K} \mbb{F}^S = 
			\left(
			\begin{array}{ccc}
				\mbf{F}_{AA} & \frac{1}{4} \mbf{F}_{Aa} & 0 \\
				0 & \frac{1}{2} \mbf{F}_{Aa} & 0 \\
				0 & \frac{1}{4} \mbf{F}_{Aa} & \mbf{F}_{aa}\\
			\end{array} \right),  \label{eq:BlkFS_appendix}
\end{eqnarray}
\end{linenomath*}
\noindent and 
\begin{linenomath*}
\begin{equation} \label{eq:BlkFX_appendix}
	\mcal{F}^X = \mbf{K}^{\intercal} \mbb{H}^X(\tilde{\mbf{p}}) \mbf{K} \mbb{F}^X = 
			\left(
			\begin{array}{ccc}
				q^{\prime}_{A} \mbf{F}_{AA} & \frac{1}{2} q^{\prime}_{A} \mbf{F}_{Aa} & 0 \\ 
				q^{\prime}_{a} \mbf{F}_{AA} & \frac{1}{2} \mbf{F}_{Aa} & q^{\prime}_{A} \mbf{F}_{aa}  \\ 
				0 & \frac{1}{2} q^{\prime}_{a} \mbf{F}_{Aa} & q^{\prime}_{a} \mbf{F}_{aa}  \\
			\end{array} \right),
\end{equation}
\end{linenomath*}
\newpage

\section{Boundary stability analysis}

In this section we follow closely the derivations presented in \cite{deVriesCaswell2019a} but we extended the model to include selfing which complicates the boundary stability analysis. 

We will derive the conditions for a protected polymorphism. The derivation is simplest when the population vector is ordered by genotype first, and then by whether individuals were produced through selfing or outcrossing, and finally by stage, in contrast to the ordering used in the main text (produced by selfing vs outcrossing, then genotype, then stage). 

The population vector is then given by 
\begin{equation}
	\tilde{\mbf{n}} = \left(
			\begin{array}{c}
							\mbf{n}^S_{AA} \\
							\mbf{n}^X_{AA} \\ \hline
							\mbf{n}^S_{Aa} \\
							\mbf{n}^X_{Aa} \\ \hline
							\mbf{n}^S_{aa} \\
							\mbf{n}^X_{aa} \\
			\end{array} \right).
\end{equation}
The population projection matrix $\tilde{\mbf{A}}$ consists of $3 \times 3$ blocks, which act on the genotype specific population vectors:
{
\small
\begin{align*} 
	\tilde{\mbf{A}}[\tilde{\mbf{n}}] &= \tilde{\mbf{U}} + \tilde{\mbf{F}}\\
		&= \left(
			\begin{array}{cc|cc|cc}
				\mathbf{U}^S_{AA} & \mbf{0} & \mbf{0} & \mbf{0} & \mbf{0} & \mbf{0} \\ 
				\mbf{0} & \mathbf{U}^X_{AA} & \mbf{0} & \mbf{0} & \mbf{0} & \mbf{0} \\\hline
				\mbf{0} & \mbf{0} & \mathbf{U}^S_{Aa} & \mbf{0} & \mbf{0} & \mbf{0} \\
				\mbf{0} & \mbf{0} & \mbf{0} & \mathbf{U}^X_{Aa} & \mbf{0} &  \mbf{0}\\\hline
				\mbf{0} & \mbf{0} & \mbf{0} & \mbf{0} & \mathbf{U}^S_{aa} & \mbf{0} \\ 
				\mbf{0} & \mbf{0} & \mbf{0} & \mbf{0} & \mbf{0} & \mathbf{U}^X_{aa}
			\end{array} \right) \\ 
		&+ \left(
			\begin{array}{cc|cc|cc}
				C_{AA} (1 - \delta) \mbf{F}_{AA} & C_{AA} (1 - \delta) \mbf{F}_{AA}  & \frac{1}{4} C_{Aa} (1 - \delta) \mbf{F}_{Aa} & \frac{1}{4} C_{Aa} (1 - \delta) \mbf{F}_{Aa} & \mbf{0} & \mbf{0} \\ 
				q^{\prime}_{A}  (1 - C_{AA}) \mbf{F}_{AA}& q^{\prime}_{A} (1 - C_{AA}) \mbf{F}_{AA} & \frac{1}{2} q^{\prime}_{A} (1 - C_{Aa})\mbf{F}_{Aa}  & \frac{1}{2} q^{\prime}_{A} (1 - C_{Aa})\mbf{F}_{Aa}  & \mbf{0} & \mbf{0} \\ \hline
				\mbf{0} & \mbf{0} & \frac{1}{2} C_{Aa} (1 - \delta) \mbf{F}_{Aa} & \frac{1}{2} C_{Aa} (1 - \delta) \mbf{F}_{Aa}& \mbf{0} & \mbf{0} \\
				q^{\prime}_{a} (1 - C_{AA}) \mbf{F}_{AA} & q^{\prime}_{a}(1 - C_{AA})  \mbf{F}_{AA} & \frac{1}{2}  (1 - C_{Aa}) \mbf{F}_{Aa}& \frac{1}{2} (1 - C_{Aa})\mbf{F}_{Aa} & q^{\prime}_{A} (1 - C_{aa})\mbf{F}_{aa}  & q^{\prime}_{A}  (1 - C_{aa})\mbf{F}_{aa}\\ \hline
				\mbf{0} & \mbf{0} & \frac{1}{4} C_{Aa} (1 - \delta) \mbf{F}_{Aa} & \frac{1}{4} C_{Aa} (1 - \delta) \mbf{F}_{Aa} & C_{aa} (1 - \delta) \mbf{F}_{aa} &  C_{aa} (1 - \delta) \mbf{F}_{aa}\\ 
				\mbf{0} & \mbf{0} & \frac{1}{2} q^{\prime}_{a}  (1 - C_{Aa}) \mbf{F}_{Aa} & \frac{1}{2} q^{\prime}_{a}  (1 - C_{Aa})\mbf{F}_{Aa}& q^{\prime}_{a}(1 - C_{aa})  \mbf{F}_{aa} & q^{\prime}_{a} (1 - C_{aa})\mbf{F}_{aa} 
			\end{array} \right) \numberthis \label{eq:AtildeCoexistence}
\end{align*}
}



The Mendelian matrix model defined in equations \textcolor{red}{(9)-(13)}, or (\ref{eq:AtildeCoexistence}), reduces to a linear matrix model on the boundary where the population is initially fixed for the $A$ allele (since $q^{\prime}_{A} = 1$ and $q^{\prime}_{a} = 0$). Provided the initial population contains a nonzero number of females, the population will grow or shrink exponentially after converging to a stable population structure (see Caswell (2001), section 4.5.2.1). We can take advantage of the homogeneity of $\tilde{\mbf{F}}$ by rewriting the model in terms of the normalized population vector (the frequency vector):
\begin{equation} 
	\tilde{\mbf{p}}(t + 1) = \frac{ \tilde{\mbf{A}}[\tilde{\mbf{p}}(t)] \tilde{\mbf{p}}(t) }{ \| \tilde{\mbf{A}}[\tilde{\mbf{p}}(t)] \tilde{\mbf{p}}(t) \|},\label{eq:freq_model}
\end{equation}

\noindent where $\| \mbf{a} \|$ denotes the $1$-norm of the vector $\mbf{a}$, defined as the sum of the absolute values of the entries of the vector $\mbf{a}$. Equilibrium solutions, denoted by $\hat{\mbf{p}}$, satisfy

\begin{equation} 
	\hat{\mbf{p}} = \frac{ \tilde{\mbf{A}}[\hat{\mbf{p}}] \hat{\mbf{p}} }{ \mbf{1}^{\intercal}_{2 \omega g} \tilde{\mbf{A}}[\hat{\mbf{p}}] \hat{\mbf{p}} },\label{eq:freq_equilibrium}
\end{equation}

\noindent where the one norm can be replaced by $\mbf{1}^{\intercal}_{2 \omega g} \tilde{\mbf{A}}[\hat{\mbf{p}}] \hat{\mbf{p}}$ because $\hat{\mbf{p}}$ is nonnegative.


\subsection{Linearization at the boundary equilibrium}

The stability of the boundary equilibrium to invasions by the other allele is determined by the magnitude of the largest eigenvalue of the Jacobian matrix of the frequency model (equation (\ref{eq:freq_model})) evaluated at the equilibrium defined by equation (\ref{eq:freq_equilibrium}). If the magnitude of this eigenvalue is larger than one, then the equilibrium is unstable. The Jacobian matrix,

\begin{equation} \label{eq:genJacobian}
	\mbf{M} =  \frac{ \text{d} \tilde{\mbf{p}}(t + 1) }{ \text{d} \tilde{\mbf{p}}(t) } \bigg\rvert_{\hat{\mbf{p}}},
\end{equation}

\noindent is obtained by differentiating equation (\ref{eq:genJacobian}) and evaluating the resulting derivatives at the boundary equilibrium. This requires a long series of matrix calculus operations, and repeatedly takes advantage of the fact that $\hat{\mbf{p}}$ at the boundary contains zeros for the blocks corresponding to $Aa$ and $aa$ genotypes.
\newpage
For notational convenience, first define a matrix $\mbf{B}$ as

\begin{equation}
	\mbf{B}[\tilde{\mbf{p}}] = \frac{ \tilde{\mbf{A}}[\tilde{\mbf{p}}] }{ \mbf{1}^{\intercal}_{2 \omega g} \tilde{\mbf{A}}[\tilde{\mbf{p}}] \tilde{\mbf{p}}}, 
\end{equation}

\noindent such that equation (\ref{eq:freq_model}) can be rewritten as

\begin{equation} 
	\tilde{\mbf{p}}(t + 1) = \mbf{B}[\tilde{\mbf{p}}(t)]\tilde{\mbf{p}}(t).
\end{equation}

\noindent We differentiate to obtain

\begin{equation} 
	\text{d} \tilde{\mbf{p}}(t + 1) = \mbf{B}\text{d} \tilde{\mbf{p}}(t) + \left(\text{d} \tilde{\mbf{B}} \right) \tilde{\mbf{p}}(t),
\end{equation}

\noindent where the explicit dependence of $\mbf{B}$ on $\tilde{\mbf{p}}$ has been omitted to avoid a cluttering of brackets. Multiply the second term by a $2 \omega g \times 2 \omega g$ identity matrix,

\begin{equation} 
	\text{d} \tilde{\mbf{p}}(t + 1) = \mbf{B}\text{d} \tilde{\mbf{p}}(t) + \mbf{1}_{2 \omega g} \left(\text{d} \tilde{\mbf{B}} \right) \tilde{\mbf{p}}(t).
\end{equation}

\noindent and apply the vector operator to both sides, remembering that as $\tilde{\mbf{p}}$ is a vector, $\text{vec}\tilde{\mbf{p}} = \tilde{\mbf{p}}$,

\begin{equation} 
	\text{d} \tilde{\mbf{p}}(t + 1) = \mbf{B}\text{d} \tilde{\mbf{p}}(t) + \text{vec} \left[ \mbf{1}_{2 \omega g} \left(\text{d} \tilde{\mbf{B}} \right) \tilde{\mbf{p}}(t) \right].
\end{equation}

\noindent Next, we can apply Roth's theorem (Roth 1934), $\text{vec}\mbf{ABC} = \left( \mbf{C}^{\intercal} \otimes \mbf{A} \right) \text{vec}\mbf{B}$, to replace the the $\text{vec}$ operator with the Kronecer product:

\begin{equation} 
	\text{d} \tilde{\mbf{p}}(t + 1) = \mbf{B}\text{d} \tilde{\mbf{p}}(t) + \left( \tilde{\mbf{p}}^{\intercal}(t) \otimes \mbf{1}_{2 \omega g} \right) \text{d}\mbox{vec} \mbf{B}.
\end{equation}

\noindent Then the first identification theorem and the chain rule together give the following formula for the Jacobian (Magnus and Neudecker, 1985; Caswell, 2007),

\begin{align*} \label{eq:genJacobian}
	\mbf{M} &=  \frac{ \text{d} \tilde{\mbf{p}}(t + 1) }{ \text{d} \tilde{\mbf{p}}(t) } \bigg\rvert_{\hat{\mbf{p}}}, \\
			&= \mbf{B}[\tilde{\mbf{p}}] + \left( \tilde{\mbf{p}}^{\intercal} \otimes \mbf{1}_{2 \omega g} \right) \frac{ \partial \text{vec}{\mbf{B}[\tilde{\mbf{p}}]} }{ \partial \tilde{\mbf{p}}^{\intercal} } \bigg\rvert_{\hat{\mbf{p}}},
\end{align*}
\newpage
Our aim is to express the Jacobian matrix $\mbf{M}$ in terms of the genotype specific matrices, $\mbf{U}^S_i$, $\mbf{U}^X_i$, $\mbf{F}^S_i$, $\mbf{F}^X_i$. We analyze the Jacobian at the $AA$ boundary; the expression at the $aa$ boundary can be derived afterwards using symmetry arguments. First it will be convenient to define the scalar $f(\tilde{\mbf{p}})$ as

\begin{equation}
	f(\tilde{\mbf{p}}) = \frac{1}{ \mbf{1}^{\intercal}_{2 \omega g} \tilde{\mbf{A}}[\tilde{\mbf{p}}] \tilde{\mbf{p}}},
\end{equation}

\noindent so that

\begin{equation}
	\mbf{B}[\tilde{\mbf{p}}] = f(\tilde{\mbf{p}}) \tilde{\mbf{A}}[\tilde{\mbf{p}}]. \label{eq:BfandA}
\end{equation}

\noindent Note that where it does not create confusion, we omit the explicit dependence on $\tilde{\mbf{A}}$, $\mbf{B}$, and $f$ on $\tilde{\mbf{p}}$. Next, we take the vec of both sides of equation (\ref{eq:BfandA}) and differentiate to obtain

\begin{equation} \label{eq:dvecB}
	\text{dvec} \mbf{B} = \text{vec} \tilde{\mbf{A}} \text{d} f + f \text{dvec} \tilde{\mbf{A}},
\end{equation}

\noindent or

\begin{equation}
	\frac{\partial \text{vec} \mbf{B}} {\partial \tilde{\mbf{p}}^{\intercal}} = \text{vec} \tilde{\mbf{A}} \frac{\partial f}{\partial \tilde{\mbf{p}}^{\intercal} } + f(\tilde{\mbf{p}}) \frac{ \partial \text{vec} \tilde{\mbf{A}} }{ \partial \tilde{\mbf{p}}^{\intercal} }.
\end{equation}

\noindent Next differentiate $f$ to obtain

\begin{equation}
	\text{d} f = \frac{ -1 }{ \left( \mbf{1}^{\intercal}_{2 \omega g} \tilde{\mbf{A}}[\tilde{\mbf{p}}] \tilde{\mbf{p}} \right)^2 } \left[ \mbf{1}^{\intercal}_{2 \omega g} \left( \text{d} \tilde{\mbf{A}} \right) \tilde{\mbf{p}} + \mbf{1}^{\intercal}_{2 \omega g} \tilde{\mbf{A}} \text{d} \tilde{\mbf{p}} \right].
\end{equation}

\noindent All the terms in the Jacobian are evaluated at the $AA$ boundary, which simplifies some of the equations, e.g., $\tilde{\mbf{A}}[ \hat{\mbf{p}} ] \hat{\mbf{p}} = \lambda_{AA} \hat{\mbf{p}}$, and therefore 

\begin{equation} \label{eq:lambdaAA}
	\mbf{1}^{\intercal}_{2 \omega g} \tilde{\mbf{A}}[\hat{\mbf{p}}] \hat{\mbf{p}} = \lambda_{AA}.
\end{equation}

Evaluate the differential of $f$ at the boundary and use equation (\ref{eq:lambdaAA}) to obtain

\begin{equation} \label{eq:df}
	\text{d} f(\tilde{\mbf{p}}) = \frac{ -1 }{\lambda_{AA}^2 } \left[ \mbf{1}^{\intercal}_{2 \omega g} \left( \text{d} \tilde{\mbf{A}} \right) \hat{\mbf{p}} + \mbf{1}^{\intercal}_{2 \omega g} \tilde{\mbf{A}} \text{d} \tilde{\mbf{p}} \right] \bigg\rvert_{\hat{\mbf{p}}}.
\end{equation}

\newpage

The first term in this sum, $\mbf{1}^{\intercal}_{2 \omega g} \left( \text{d} \tilde{\mbf{A}} \right) \hat{\mbf{p}}$, is equal to zero because
\begin{align*}
	q^{\prime}_A + q^{\prime}_a &= 1 \\
	\text{d}q^{\prime}_A + \text{d}q^{\prime}_a &= 0, 
\end{align*}
and therefore every column in $\left( \text{d} \tilde{\mbf{A}} \right)$ sums to zero, see equation (\ref{eq:AtildeCoexistence}).

Substituting equation (\ref{eq:df}) into equation (\ref{eq:dvecB}) and evaluating at the boundary yields
\begin{equation} \label{eq:dvecBsubs}
	\text{dvec} \mbf{B} = \frac{ -1 }{\lambda_{AA}^2 } \text{vec} \tilde{\mbf{A}} \left[ \mbf{1}^{\intercal}_{2 \omega g} \tilde{\mbf{A}} \text{d} \tilde{\mbf{p}} \right] + \frac{ 1 }{\lambda_{AA} } \text{dvec} \tilde{\mbf{A}},
\end{equation}
\noindent or 
\begin{equation}
	\frac{\partial \text{vec} \mbf{B}} {\partial \tilde{\mbf{p}}^{\intercal}} \bigg\rvert_{\hat{\mbf{p}}} = \frac{ -1 }{\lambda_{AA}^2 } \left( \text{vec} \tilde{\mbf{A}} \right) \left( \mbf{1}^{\intercal}_{2 \omega g} \tilde{\mbf{A}} \text{d} \tilde{\mbf{p}} \right) + \frac{ 1 }{\lambda_{AA} } \left. \frac{\partial \text{vec} \mbf{A}} {\partial \tilde{\mbf{p}}^{\intercal}} \right|_{\hat{\mbf{p}}}
\end{equation}

\noindent Finally, substituting the expression above into equation (\ref{eq:genJacobian}) yields the Jacobian matrix:

\begin{align*} \label{eq:subJacobian}
	\mbf{M} &=  \mbf{B}[\tilde{\mbf{p}}] + \left( \tilde{\mbf{p}}^{\intercal} \otimes \mbf{1}_{2 \omega g} \right) \frac{ \partial \text{vec}{\mbf{B}} }{ \partial \tilde{\mbf{p}}^{\intercal} } \bigg\rvert_{\hat{\mbf{p}}}, \\
			&= \underbrace{ \mbf{B}[\hat{\mbf{p}}]}_{\circled{A}} - 
			   \underbrace{ \frac{ 1 }{\lambda_{AA}^2 } (\hat{\mbf{p}}^{\intercal} \otimes \mbf{I}_{\omega g}) \left( \text{vec} \tilde{\mbf{A}} \right) \left( \mbf{1}^{\intercal}_{2 \omega g} \tilde{\mbf{A}} \right)}_{\circled{B}} \\
			&~~~~~~~~~~~+ \underbrace{\frac{ 1 }{\lambda_{AA} } (\hat{\mbf{p}}^{\intercal} \otimes \mbf{I}_{\omega g}) \left. \frac{\partial \text{vec} \tilde{\mbf{A}}}{\partial \tilde{\mbf{p}}^{\intercal}} \right|_{\hat{\mbf{p}}}}_{\circled{C}},
\end{align*}

\newpage
%%%%%%%%%%%%%%%%
\subsection{Components of the Jacobian}

The next task is to work out each of the three terms \circled{A}, \circled{B}, and \circled{C}, in $\mbf{M}$. Beginning with \circled{A}, we have

\begin{align*}
	\circled{A}&=\mbf{B}[\hat{\mbf{p}}] = \frac{ \tilde{\mbf{A}}[\hat{\mbf{p}}] }{ \mbf{1}^{\intercal}_{\omega g} \tilde{\mbf{A}}[\hat{\mbf{p}}] \hat{\mbf{p}} } = \\
						   \frac{1}{\lambda_{AA}} &
						   \left(
			\begin{array}{cc|cc|cc}
				\mathbf{U}^S_{AA} & \mbf{0}& \mbf{0} & \mbf{0} & \mbf{0} \\ 
			\mbf{0} & \mathbf{U}^X_{AA} & \mbf{0} & \mbf{0} & \mbf{0} & \mbf{0} \\ \hline
				\mbf{0} & \mbf{0} & \mathbf{U}^S_{Aa}& \mbf{0} & \mbf{0} & \mbf{0} \\
				\mbf{0} & \mbf{0} & \mbf{0}& \mcal{U}^X_{Aa} & \mbf{0} & \mbf{0}\\ \hline
				\mbf{0} & \mbf{0} & \mbf{0} & \mbf{0} & \mathbf{U}^S_{aa}  & \mbf{0} \\ 
				\mbf{0} & \mbf{0} & \mbf{0}  & \mbf{0} & \mbf{0} & \mathbf{U}^X_{aa}
			\end{array} \right) \\
			 +\frac{1}{\lambda_{AA}} &
						   \left(
			\begin{array}{cc|cc|cc}
		 C_{AA} (1 - \delta) \mbf{F}_{AA} & C_{AA} (1 - \delta) \mbf{F}_{AA}& \frac{1}{4} C_{Aa} (1 - \delta) \mbf{F}_{Aa} & \frac{1}{4} C_{Aa} (1 - \delta) \mbf{F}_{Aa}& \mbf{0} & \mbf{0} \\ 
				(1 - C_{AA}) \mbf{F}_{AA} & (1 - C_{AA}) \mbf{F}_{AA}  & \frac{1}{2}(1 - C_{Aa}) \mbf{F}_{Aa}  & \frac{1}{2} (1 - C_{Aa})\mbf{F}_{Aa} & \mbf{0} & \mbf{0} \\ \hline
				\mbf{0} & \mbf{0} & \frac{1}{2} C_{Aa} (1 - \delta) \mbf{F}_{Aa} & \frac{1}{2} C_{Aa} (1 - \delta) \mbf{F}_{Aa}  & \mbf{0} & \mbf{0} \\
				\mbf{0} & \mbf{0} & \frac{1}{2} (1 - C_{Aa}) \mbf{F}_{Aa} & \frac{1}{2} (1 - C_{Aa}) \mbf{F}_{Aa} & (1 - C_{aa}) \mbf{F}_{aa} & (1 - C_{aa}) \mbf{F}_{aa}\\ \hline
				\mbf{0} & \mbf{0} & \frac{1}{4} C_{Aa} (1 - \delta) \mbf{F}_{Aa} & \frac{1}{4} C_{Aa} (1 - \delta) \mbf{F}_{Aa}  &C_{aa} (1 - \delta) \mbf{F}_{aa} &C_{aa} (1 - \delta) \mbf{F}_{aa} \\ 
				\mbf{0} & \mbf{0} & \mbf{0}  & \mbf{0} & \mbf{0} &\mbf{0}
			\end{array} \right) 
			\numberthis
\end{align*}
Next, we have \circled{B},
\begin{equation}
	\circled{B} = - \frac{1}{\lambda_{AA}^2} \left( \hat{\mbf{p}}^{\intercal} \otimes \mbf{1}_{2 \omega g} \right) \left( \text{vec} \tilde{\mbf{A}} \right) \left( \mbf{1}^{\intercal}_{2 \omega g} \tilde{\mbf{A}} \right).
\end{equation}
Again using Roth's theorem (Roth 1934), $\left( \mbf{C}^{\intercal} \otimes \mbf{A} \right) \text{vec}\mbf{B} = \text{vec}\mbf{ABC}$, we can simplify as follows:
\begin{align*}
	\left( \hat{\mbf{p}}^{\intercal} \otimes \mbf{1}_{2 \omega g} \right) \left( \text{vec} \tilde{\mbf{A}}[\hat{\mbf{p}}] \right) &= \text{vec} \left( \mbf{I}_{2 \omega g}\, \tilde{\mbf{A}}[\hat{\mbf{p}}]\, \hat{\mbf{p}} \right) \\
					   &= \lambda_{AA} \hat{\mbf{p}}, \numberthis
\end{align*}
so that 
\begin{equation}
	\circled{B} = - \frac{1}{\lambda_{AA}} \hat{\mbf{p}} \left( \mbf{1}^{\intercal}_{2 \omega g} \tilde{\mbf{A}}[\hat{\mbf{p}}] \right).
\end{equation}

\newpage
Substitute the population vector at the $AA$ boundary,
\begin{equation}
	\hat{\mbf{p}} = \left(
			\begin{array}{c}
							\hat{\mbf{p}}^S_{AA} \\
							\hat{\mbf{p}}^X_{AA} \\ \hline
							0                  \\
							0                  \\ \hline
							0                  \\
							0                  \\
			\end{array} \right)
\end{equation}
and rewrite in terms of the block matrices to obtain the big expression on the next page 
\begin{landscape}
{\footnotesize
\begin{align*}
	\circled{B} =& \\
	&-\frac{1}{\lambda_{AA}} \left(
			\begin{array}{cc|cc|cc}
				\hat{\mbf{p}}_{AA} \otimes \mbf{1}^{\intercal}_{\omega} \mbf{U}^S_{AA} & \hat{\mbf{p}}_{AA} \otimes \mbf{1}^{\intercal}_{\omega} \mbf{U}^X_{AA} & \hat{\mbf{p}}_{AA} \otimes \mbf{1}^{\intercal}_{\omega} \mbf{U}^S_{Aa} & \hat{\mbf{p}}_{AA} \otimes \mbf{1}^{\intercal}_{\omega} \mbf{U}^X_{Aa} & \hat{\mbf{p}}_{AA} \otimes \mbf{1}^{\intercal}_{\omega} \mbf{U}^S_{aa} & \hat{\mbf{p}}_{AA} \otimes \mbf{1}^{\intercal}_{\omega} \mbf{U}^X_{aa} \\ 
				\hat{\mbf{p}}_{AA} \otimes \mbf{1}^{\intercal}_{\omega} \mbf{U}^S_{AA} & \hat{\mbf{p}}_{AA} \otimes \mbf{1}^{\intercal}_{\omega} \mbf{U}^X_{AA} & \hat{\mbf{p}}_{AA} \otimes \mbf{1}^{\intercal}_{\omega} \mbf{U}^S_{Aa} & \hat{\mbf{p}}_{AA} \otimes \mbf{1}^{\intercal}_{\omega} \mbf{U}^X_{Aa} & \hat{\mbf{p}}_{AA} \otimes \mbf{1}^{\intercal}_{\omega} \mbf{U}^S_{aa} & \hat{\mbf{p}}_{AA} \otimes \mbf{1}^{\intercal}_{\omega} \mbf{U}^X_{aa} \\ \hline
				\mbf{0} & \mbf{0} & \mbf{0} & \mbf{0} & \mbf{0} & \mbf{0} \\
				\mbf{0} & \mbf{0} & \mbf{0} & \mbf{0} & \mbf{0} & \mbf{0} \\ \hline
				\mbf{0} & \mbf{0} & \mbf{0} & \mbf{0} & \mbf{0} & \mbf{0} \\ 
				\mbf{0} & \mbf{0} & \mbf{0} & \mbf{0} & \mbf{0} & \mbf{0} 
			\end{array} \right) \\
	&-\frac{1}{\lambda_{AA}} \left(
			\begin{array}{cc|cc|cc}
				(1 - \delta C_{AA}) \hat{\mbf{p}}_{AA} \otimes \mbf{1}^{\intercal}_{\omega} \mbf{F}_{AA} & (1 - \delta C_{AA}) \hat{\mbf{p}}_{AA} \otimes \mbf{1}^{\intercal}_{\omega} \mbf{F}_{AA} & (1 - \delta C_{Aa}) \hat{\mbf{p}}_{AA} \otimes \mbf{1}^{\intercal}_{\omega} \mbf{F}_{Aa} & (1 - \delta C_{Aa}) \hat{\mbf{p}}_{AA} \otimes \mbf{1}^{\intercal}_{\omega} \mbf{F}_{Aa}  & (1 - \delta C_{aa}) \hat{\mbf{p}}_{AA} \otimes \mbf{1}^{\intercal}_{\omega} \mbf{F}_{aa} & (1 - \delta C_{aa}) \hat{\mbf{p}}_{AA} \otimes \mbf{1}^{\intercal}_{\omega} \mbf{F}_{aa} \\ 
				 	(1 - \delta C_{AA}) \hat{\mbf{p}}_{AA} \otimes \mbf{1}^{\intercal}_{\omega} \mbf{F}_{AA} & 	(1 - \delta C_{AA}) \hat{\mbf{p}}_{AA} \otimes \mbf{1}^{\intercal}_{\omega} \mbf{F}_{AA} & (1 - \delta C_{Aa})  \hat{\mbf{p}}_{AA} \otimes \mbf{1}^{\intercal}_{\omega} \mbf{F}_{Aa} & (1 - \delta C_{Aa})  \hat{\mbf{p}}_{AA} \otimes \mbf{1}^{\intercal}_{\omega} \mbf{F}_{Aa} & (1 - \delta C_{aa}) \hat{\mbf{p}}_{AA} \otimes \mbf{1}^{\intercal}_{\omega} \mbf{F}_{aa} & (1 - \delta C_{aa}) \hat{\mbf{p}}_{AA} \otimes \mbf{1}^{\intercal}_{\omega} \mbf{F}_{aa}\\ \hline
				\mbf{0} & \mbf{0} & \mbf{0} & \mbf{0} & \mbf{0} & \mbf{0} \\
				\mbf{0} & \mbf{0} & \mbf{0} & \mbf{0} & \mbf{0} & \mbf{0} \\ \hline
				\mbf{0} & \mbf{0} & \mbf{0} & \mbf{0} & \mbf{0} & \mbf{0} \\ 
				\mbf{0} & \mbf{0} & \mbf{0} & \mbf{0} & \mbf{0} & \mbf{0}
			\end{array} \right) \\
			 \numberthis			
\end{align*}
}

\end{landscape}


Finally, we turn to term \circled{C} of the Jacobian. 
To derive term $\circled{C}$ in the Jacobian, we first derive a useful expression for $\text{vec} \tilde{\bo A}$ in terms of its component block matrices. The matrix $\tilde{\mathbf{A}}$ can be decomposed into 36 $\omega \times \omega$ block matrices, as in equation (\ref{eq:AtildeCoexistence}), so that for example 
\begin{eqnarray}
\mathbf{A}_{11}&=&  \mathbf{U}^S_{AA} + C_{AA} (1 - \delta) \mbf{F}_{AA} ,\\
\mathbf{A}_{22}&=&\mathbf{U}^X_{AA} +q^{\prime}_{A} (1 - C_{AA}) \mbf{F}_{AA},
\end{eqnarray}
and
\begin{eqnarray}
  \mathbf{A}_{21}&=&q^{\prime}_{a} (1 - C_{AA}) \mbf{F}_{AA},\\
  \mathbf{A}_{12}&=& C_{AA} (1 - \delta) \mbf{F}_{AA} .
\end{eqnarray}
The matrix $\tilde{\mathbf{A}}$ can then be written as 
\begin{eqnarray}
\tilde{\mathbf{A}}&=&\sum_{i,j=1}^6 \mathbf{E}_{ij} \otimes \mathbf{A}_{ij}, \\
&=&\sum_{i,j=1}^6 \left(\mathbf{e}_i \mathbf{e}^\tr_j\right) \otimes \left(\mathbf{A}_{ij} \mathbf{I}_\omega \right),
\label{eq:A_sum_appendix}
\end{eqnarray}
where we have used the definition of the matrix $\mathbf{E}_{ij}=\mathbf{e}_i \mathbf{e}^\tr_j$. Using $\mathbf{A} \mathbf{C}\otimes \mathbf{B} \mathbf{D}=(\mathbf{A} \otimes \mathbf{B}) (\mathbf{C} \otimes \mathbf{D})$, equation (\ref{eq:A_sum_appendix}) can be rewritten as
\begin{equation}
  \tilde{\mathbf{A}}=\sum_{i,j=1}^6 \left(\mathbf{e}_i \otimes \mathbf{A}_{ij} \right) \left( \mathbf{e}^\tr_j \otimes \mathbf{I}_\omega \right) ,
\end{equation}
next use the identity $ \sum_i \left( \mathbf{e}_i \otimes \mathbf{I}_\omega \right) \mathbf{A}_{ij} = \sum_i \mathbf{e}_i \otimes \mathbf{A}_{ij}$ to rewrite this again 
\begin{eqnarray}
\tilde{\mathbf{A}}=\sum_{i,j=1}^6 \left(\mathbf{e}_i \otimes  \mathbf{I}_\omega \right) \mathbf{A}_{ij} \left( \mathbf{e}^\tr_j \otimes \mathbf{I}_\omega \right).
  \end{eqnarray}
Use Roth's theorem, $\text{vec}\mathbf{A}\mathbf{B}\mathbf{C}=\left(\mathbf{C}^\tr \otimes \mathbf{A}\right) \text{vec} \mathbf{B}$, where $\mathbf{A}=\left(\mathbf{e}_i \otimes  \mathbf{I}_\omega \right)$, $\mathbf{B}=\mathbf{A}_{ij}$ and $\mathbf{C}=\left( \mathbf{e}^\tr_j \otimes \mathbf{I}_\omega \right)$, to obtain the following formula for  $\mbox{vec}\mathbf{A}$:
\begin{equation}
\mbox{vec}\tilde{\mathbf{A}}=\sum_{i,j}^6 \left(\mathbf{e}_j \otimes \mathbf{I}_\omega\right) \otimes \left(\mathbf{e}_i \otimes \mathbf{I}_\omega\right)\mbox{vec}\mathbf{A}_{ij}. \label{eq:vecAinAijs}
\end{equation}

Now we are ready to analyze term $\circled{C}$ in the Jacobian. Replace the derivative of $\mbox{vec} \tilde{\mathbf{A}}$ with equation (\ref{eq:vecAinAijs}), such that 
\begin{eqnarray}
  \frac{1}{\lambda_{AA}}(\hat{\bo p}^\tr \otimes  \mathbf{I}_{2\omega g})\frac{\partial \mbox{vec} \mathbf{A}}{\partial \tilde{\bo p}^\tr}\bigg\rvert_{\hat{\bo p}}&=&\frac{1}{\lambda_{AA}}\sum_{i,j=1}^{6} \left(\hat{\bo p}^\tr \otimes \mathbf{I}_{2\omega g}\right)\left(\left(\mathbf{e}_{j} \otimes \mathbf{I}_\omega\right) \otimes \left(\mathbf{e}_{i} \otimes \mathbf{I}_\omega\right) \right)\frac{\partial \mbox{vec} \mathbf{A}_{ij}}{\partial \tilde{\bo p}^\tr}\bigg\rvert_{\hat{\bo p}}.\nonumber\\ \label{eq:third_term_J}
\end{eqnarray}
Use $(\mathbf{A} \otimes \mathbf{B}) (\mathbf{C} \otimes \mathbf{D})= \mathbf{A} \mathbf{C}\otimes \mathbf{B} \mathbf{D}$ to rewrite  
\begin{equation}
  \left(\hat{\bo p}^\tr \otimes  \mathbf{I}_{2\omega g}\right) \left(\left(\mathbf{e}_{j} \otimes \mathbf{I}_\omega\right) \otimes \left(\mathbf{e}_{i} \otimes \mathbf{I}_\omega\right)\right)= \left(\hat{\bo p}^\tr \left(\mathbf{e}_{j} \otimes \mathbf{I}_\omega\right) \right) \otimes  \left(\mathbf{I}_{2\omega g} (\mathbf{e}_{i} \otimes \mathbf{I}_\omega) \right), \label{eq:rewrite_kron_four_matrices}
\end{equation}
substituting this expression into the right hand side of equation (\ref{eq:third_term_J}) yields
\begin{eqnarray}
\frac{1}{\lambda_{AA}}(\hat{\mathbf{p}}^\tr \otimes \mathbf{I}_{2\omega g})\frac{\partial \mbox{vec} \mathbf{A}}{\partial \tilde{\bo p}^\tr}\bigg\rvert_{\hat{\bo p}}&=&\frac{1}{\lambda_{AA}}\sum_{i,j=1}^{6} \left(\hat{\bo p}^\tr \left(\mathbf{e}_{j} \otimes \mathbf{I}_\omega\right)\right)\otimes \left( \mathbf{I}_{2\omega g} \left(\mathbf{e}_{i} \otimes \mathbf{I}_\omega\right)\right) \frac{\partial \mbox{vec} \mathbf{A}_{ij}}{\partial \tilde{\bo p}^\tr}\bigg\rvert_{\hat{\bo p}}.\nonumber\\ \label{eq:45_app}
\end{eqnarray}


Next substitute $\hat{\mbf{p}}^{\intercal} = \left( \hat{\mbf{p}}^{\intercal}_{S,AA},\hat{\mbf{p}}^{\intercal}_{X,AA},\mbf{0},\mbf{0},\mbf{0},\mbf{0} \right)$ into the right side of equation (\ref{eq:45_app}) so that only terms with $j=1$ and $j=2$ are nonzero, yielding

\begin{align*}
	\frac{1}{\lambda_{AA}}(\hat{\mbf{p}}^{\intercal} \otimes \mbf{I}_{2 \omega g}) \left. \frac{ \partial \text{vec}\mbf{A}}{\partial \tilde{\mbf{p}}^{\intercal}} \right|_{\hat{\mbf{p}}} &= \frac{1}{\lambda_{AA}} \sum^{6}_{i=1} \big( \hat{\mbf{p}}^{\intercal}_{S,AA} \otimes (\mbf{e}_{i} \otimes \mbf{I}_{\omega}) \big) \left. \frac{ \partial \text{vec}\mbf{A}_{i,1}}{\partial \tilde{\mbf{p}}^{\intercal}} \right|_{\hat{\mbf{p}}} \\
		&+ \frac{1}{\lambda_{AA}} \sum^{6}_{i=1} \big( \hat{\mbf{p}}^{\intercal}_{X,AA} \otimes (\mbf{e}_{i} \otimes \mbf{I}_{\omega}) \big) \left. \frac{ \partial \text{vec}\mbf{A}_{i,2}}{\partial \tilde{\mbf{p}}^{\intercal}} \right|_{\hat{\mbf{p}}} \numberthis
\end{align*}

\noindent Selfing does not depend on the frequency vector, and therefore none of the $\mbf{A}_{1,i}$ are a function of the frequency vector $\mbf{p}$,

\begin{equation}
	\left. \frac{ \partial \text{vec}\mbf{A}_{1,i}}{\partial \tilde{\mbf{p}}^{\intercal}} \right|_{\hat{\mbf{p}}} = 0,\,\text{for all }i. 
\end{equation}

\noindent (see Eq{\ref{eq:AtildeCoexistence}}). Next, write down each term in the sum over $i$ and take the derivative  of the $\text{vec}\mbf{A}_{i,1}$'s to obtain
\begin{align*}
	\frac{1}{\lambda_{AA}}(\hat{\mbf{p}}^{\intercal} \otimes \mbf{I}_{2 \omega g}) \left. \frac{ \partial \text{vec}\mbf{A}}{\partial \tilde{\mbf{p}}^{\intercal}} \right|_{\hat{\mbf{p}}} &= \frac{1}{\lambda_{AA}} \big( \hat{\mbf{p}}^{\intercal}_{S,AA} \otimes (\mbf{e}_{2} \otimes \mbf{I}_{\omega}) \big) \left. \frac{ \partial \text{vec}\mbf{A}_{21}}{\partial \tilde{\mbf{p}}^{\intercal}} \right|_{\hat{\mbf{p}}}, \\
		&+ \frac{1}{\lambda_{AA}} \big( \hat{\mbf{p}}^{\intercal}_{X,AA} \otimes (\mbf{e}_{2} \otimes \mbf{I}_{\omega}) \big) \left. \frac{ \partial \text{vec}\mbf{A}_{22}}{\partial \tilde{\mbf{p}}^{\intercal}} \right|_{\hat{\mbf{p}}}, \\
		&+\frac{1}{\lambda_{AA}} \big( \hat{\mbf{p}}^{\intercal}_{S,AA} \otimes (\mbf{e}_{4} \otimes \mbf{I}_{\omega}) \big) \left. \frac{ \partial \text{vec}\mbf{A}_{41}}{\partial \tilde{\mbf{p}}^{\intercal}} \right|_{\hat{\mbf{p}}}, \\
		&+ \frac{1}{\lambda_{AA}} \big( \hat{\mbf{p}}^{\intercal}_{X,AA} \otimes (\mbf{e}_{4} \otimes \mbf{I}_{\omega}) \big) \left. \frac{ \partial \text{vec}\mbf{A}_{42}}{\partial \tilde{\mbf{p}}^{\intercal}} \right|_{\hat{\mbf{p}}}. \numberthis
\end{align*}
Substituting in the expressions for the block matrices yields
\begin{align*}
	\frac{1}{\lambda_{AA}}(\hat{\mbf{p}}^{\intercal} \otimes \mbf{I}_{2 \omega g}) \left. \frac{ \partial \text{vec}\mbf{A}}{\partial \hat{\mbf{p}}^{\intercal}} \right|_{\hat{\mbf{p}}} = 
		&\frac{(1 - C_{AA})}{\lambda_{AA}} \big[ \hat{\mbf{p}}^{\intercal}_{X,AA}+ \hat{\mbf{p}}^{\intercal}_{S,AA} \big]\otimes (\mbf{e}_{2} \otimes \mbf{I}_{\omega}) \text{vec}\mbf{F}_{AA} \left.\frac{\partial q^{\prime}_{A}}{\partial \tilde{\mbf{p}}^{\intercal}} \right|_{\hat{\mbf{p}}} \\
		&-\frac{(1 - C_{AA})}{\lambda_{AA}} \big[ \hat{\mbf{p}}^{\intercal}_{X,AA}+ \hat{\mbf{p}}^{\intercal}_{S,AA} \big]\otimes (\mbf{e}_{4} \otimes \mbf{I}_{\omega}) \text{vec}\mbf{F}_{AA} \left.\frac{\partial q^{\prime}_{A}}{\partial \tilde{\mbf{p}}^{\intercal}} \right|_{\hat{\mbf{p}}} \\
	   \numberthis
\end{align*}

\begin{landscape}
Finally, apply Roth's theorem (Roth 1934), $\left( \mbf{C}^{\intercal} \otimes \mbf{A} \right) \text{vec}\mbf{B} = \text{vec}\mbf{ABC}$ (e.g., where $\mbf{C}^{\intercal} = \hat{\mbf{p}}^{\intercal}_{AA}$, $\mbf{A} = (\mbf{e}_1 \otimes \mbf{I}_{\omega})$, and $\text{vec}\mbf{B} = \text{vec}\mbf{F}_{AA}$) to rewrite as

\begin{align*}
	\frac{1}{\lambda_{AA}}(\hat{\mbf{p}}^{\intercal} \otimes \mbf{I}_{2 \omega g}) \left. \frac{ \partial \text{vec}\mbf{A}}{\partial \hat{\mbf{p}}^{\intercal}} \right|_{\hat{\mbf{p}}} = 
		&\frac{(1 - C_{AA})}{\lambda_{AA}}  \text{vec}\big[(\mbf{e}_{2} \otimes \mbf{I}_{\omega}) \mbf{F}_{AA} \big(\hat{\mbf{p}}_{X,AA}+ \hat{\mbf{p}}_{S,AA}  \big)\big]  \left.\frac{\partial q^{\prime}_{A}}{\partial \tilde{\mbf{p}}^{\intercal}} \right|_{\hat{\mbf{p}}}  \\
		&- \frac{(1 - C_{AA})}{\lambda_{AA}}  \text{vec}\big[(\mbf{e}_{4} \otimes \mbf{I}_{\omega}) \mbf{F}_{AA} \big(\hat{\mbf{p}}_{X,AA}+ \hat{\mbf{p}}_{S,AA}  \big)\big]  \left.\frac{\partial q^{\prime}_{A}}{\partial \tilde{\mbf{p}}^{\intercal}} \right|_{\hat{\mbf{p}}}  \label{eq:partCwithpartials_app} \numberthis
\end{align*}

\noindent written in terms of the block matrices, this yields
{
\scriptsize
\begin{align*}
	\circled{C} =& \frac{(1 - C_{AA})}{\lambda_{AA}} \,* \\
	&\left(
			\begin{array}{cc|cc|cc}
					\mbf{0} & \mbf{0} & \mbf{0} & \mbf{0} & \mbf{0} & \mbf{0}  \\
			\mbf{F}_{AA} \big(\hat{\mbf{p}}_{X,AA}+ \hat{\mbf{p}}_{S,AA}  \big) \frac{\partial q^{\prime}_{A}}{\partial \mbf{p}^{ \intercal}_{S,AA}}  &  \mbf{F}_{AA} \big(\hat{\mbf{p}}_{X,AA}+ \hat{\mbf{p}}_{S,AA}  \big) \frac{\partial q^{\prime}_{A}}{\partial \mbf{p}^{ \intercal}_{X,AA}} &  \mbf{F}_{AA} \big(\hat{\mbf{p}}_{X,AA}+ \hat{\mbf{p}}_{S,AA}  \big) \frac{\partial q^{\prime}_{A}}{\partial \mbf{p}^{\intercal}_{S,Aa}}& \mbf{F}_{AA} \big(\hat{\mbf{p}}_{X,AA}+ \hat{\mbf{p}}_{S,AA}  \big) \frac{\partial q^{\prime}_{A}}{\partial \mbf{p}^{\intercal}_{X,Aa}} & \mbf{F}_{AA} \big(\hat{\mbf{p}}_{X,AA}+ \hat{\mbf{p}}_{S,AA}  \big) \frac{\partial q^{\prime}_{A}}{\partial \mbf{p}^{\intercal}_{S,aa}}&  \mbf{F}_{AA} \big(\hat{\mbf{p}}_{X,AA}+ \hat{\mbf{p}}_{S,AA}  \big) \frac{\partial q^{\prime}_{A}}{\partial \mbf{p}^{\intercal}_{X,aa}} \\ \hline
				\mbf{0} & \mbf{0} & \mbf{0} & \mbf{0} & \mbf{0} & \mbf{0} \\
				-\mbf{F}_{AA} \big(\hat{\mbf{p}}_{X,AA}+ \hat{\mbf{p}}_{S,AA}  \big) \frac{\partial q^{\prime}_{A}}{\partial \mbf{p}^{ \intercal}_{S,AA}}  &  -\mbf{F}_{AA} \big(\hat{\mbf{p}}_{X,AA}+ \hat{\mbf{p}}_{S,AA}  \big) \frac{\partial q^{\prime}_{A}}{\partial \mbf{p}^{ \intercal}_{X,AA}} & - \mbf{F}_{AA} \big(\hat{\mbf{p}}_{X,AA}+ \hat{\mbf{p}}_{S,AA}  \big) \frac{\partial q^{\prime}_{A}}{\partial \mbf{p}^{\intercal}_{S,Aa}}& -\mbf{F}_{AA} \big(\hat{\mbf{p}}_{X,AA}+ \hat{\mbf{p}}_{S,AA}  \big) \frac{\partial q^{\prime}_{A}}{\partial \mbf{p}^{\intercal}_{X,Aa}} &- \mbf{F}_{AA} \big(\hat{\mbf{p}}_{X,AA}+ \hat{\mbf{p}}_{S,AA}  \big) \frac{\partial q^{\prime}_{A}}{\partial \mbf{p}^{\intercal}_{S,aa}}& - \mbf{F}_{AA} \big(\hat{\mbf{p}}_{X,AA}+ \hat{\mbf{p}}_{S,AA}  \big) \frac{\partial q^{\prime}_{A}}{\partial \mbf{p}^{\intercal}_{X,aa}} \\ \hline
				\mbf{0} & \mbf{0} & \mbf{0} & \mbf{0} & \mbf{0} & \mbf{0} \\ 
				\mbf{0} & \mbf{0} & \mbf{0} & \mbf{0} & \mbf{0} & \mbf{0} 
			\end{array} \right)  \numberthis			
\end{align*}
}
The above expression requires the derivative of the frequency of allele $A$ in the gamete pool with respect to the population frequency vector:

\begin{equation}
	\left. \frac{\partial q^{\prime}_{A}}{\partial \tilde{\mbf{p}}^{\intercal}} \right|_{\hat{\mbf{P}}}
\end{equation}


 \end{landscape}


 Start with equation \textcolor{red}{(4)} from the main text:
\begin{linenomath*}
\begin{equation} \label{eq:maleGametePool}
	\left(
		\begin{array}{c}
			q^{\prime}_{A} \\
			q^{\prime}_{a} \\
		\end{array} \right) = 
			\frac{\mbf{W}^{\prime} \mbb{F}^{\prime} \mbf{p}^{\prime}}{\| \mbf{W}^{\prime} \mbb{F}^{\prime} \mbf{p}^{\prime} \|} = 
				\frac{\mbf{W}^{\prime} \mbb{F}^{\prime} (\mbf{n}^X + \mbf{n}^S)}{\| \mbf{W}^{\prime} \mbb{F}^{\prime} (\mbf{n}^X + \mbf{n}^S)\|},
\end{equation}
\end{linenomath*}
therefore
\begin{equation}
  q^\prime_A=\frac{\mathbf{e}_1^\tr \mathbf{W}^\prime \mathbf{F}^\prime  \mathbf{p}^\prime}{\mathbf{1}^\tr_{2}\mathbf{W}^\prime\mathbf{F}^\prime\mathbf{p}^\prime}, \label{eq:male_gamete_pool}
\end{equation}
 where the one norm can be replaced by $\mathbf{1}^\tr_{2}\mathbf{W}^\prime\mathbf{F}^\prime \mathbf{p}^\prime$ because $\mathbf{p}^\prime$ is nonnegative. For convenience, we will denote the normalizing factor in the denominator with $ \mbox{p}_{n}$, 
 \begin{equation}
   \mbox{p}_{n}=\mathbf{1}^\tr_{2}\mathbf{W}^\prime\mathbf{F}^\prime \mathbf{p}^\prime
 \end{equation}
 Taking the derivative of $ q^\prime_A$ yields
\begin{eqnarray}
  \frac{\partial   q^\prime_A }{\partial \tilde{\bo p}^\tr}&=& \frac{1}{ \mbox{p}_{n}}\mathbf{e}_1^\tr \mathbf{W}^\prime \mathbf{F}^\prime   \frac{\partial  \mathbf{p}^\prime}{\partial \tilde{\bo p}^\tr} - \frac{\mathbf{e}_1^\tr \mathbf{W}^\prime \mathbf{F}^\prime \mathbf{p}^\prime}{ \mbox{p}_{n}^2} \left( \mathbf{1}^\tr_{2}\mathbf{W}^\prime\mathbf{F}^\prime \frac{\partial  \mathbf{p}^\prime}{\partial \tilde{\bo p}^\tr} \right). \label{eq:56_app}
\end{eqnarray}
Recall 
\begin{equation}
  \mathbf{p}^\prime=\left(\begin{array} {c}
\mathbf{p}^S_{AA}+\mathbf{p}^X_{AA}\\
\mathbf{p}^S_{Aa}+\mathbf{p}^X_{Aa}\\
\mathbf{p}^S_{aa}+\mathbf{p}^X_{aa}
  \end{array}\right),
\end{equation}
and 
\begin{equation}
  \tilde{\bo p}=\left(\begin{array} {c}
\mathbf{p}_{AA}^S\\
\mathbf{p}_{AA}^X\\
\hline
\mathbf{p}_{Aa}^S\\
\mathbf{p}_{Aa}^X\\
\hline
\mathbf{p}_{aa}^S\\
\mathbf{p}_{aa}^X
  \end{array}\right),
\end{equation}
to calculate 
\begin{equation}
\frac{\partial  \mathbf{p}^\prime}{\partial \tilde{\bo p}^\tr} =  \left(\begin{array} {cccccc}
\mathbf{I}& \mathbf{I}& \mathbf{0}  &\mathbf{0} &\mathbf{0}  &\mathbf{0} \\
\mathbf{0} &\mathbf{0}  &\mathbf{I}& \mathbf{I}& \mathbf{0} &\mathbf{0} \\
\mathbf{0} &\mathbf{0}  &\mathbf{0} &\mathbf{0}  &  \mathbf{I}& \mathbf{I}
\end{array}\right).
\end{equation}
First we will evaluate the first term in the sum in equation (\ref{eq:56_app}),
\begin{eqnarray}
	 \frac{1}{ \mbox{p}_{n}}\mathbf{e}_1^\tr \mathbf{W}^\prime \mathbf{F}^\prime   \frac{\partial  \mathbf{p}^\prime}{\partial \tilde{\bo p}^\tr} &=&  \frac{1}{ \mbox{p}_{n}} (1,0) \left(\begin{array} {ccc}
\mathbf{1}^\tr_{\omega} \mathbf{F}_{AA}^\prime& \frac{1}{2}\mathbf{1}^\tr_{\omega} \mathbf{F}_{Aa}^\prime &\mathbf{0} \\
\mathbf{0} &\frac{1}{2}\mathbf{1}^\tr_{\omega} \mathbf{F}_{Aa}^\prime & \mathbf{1}^\tr_{\omega} \mathbf{F}_{aa}^\prime
\end{array}\right) \left(\begin{array} {cccccc}
\mathbf{I}& \mathbf{I}& \mathbf{0}  &\mathbf{0} &\mathbf{0}  &\mathbf{0} \\
\mathbf{0} &\mathbf{0}  &\mathbf{I}& \mathbf{I}& \mathbf{0} &\mathbf{0} \\
\mathbf{0} &\mathbf{0}  &\mathbf{0} &\mathbf{0}  &  \mathbf{I}& \mathbf{I}
\end{array}\right) \nonumber \\
&=& \frac{1}{ \mbox{p}_{n}} \left( \mathbf{1}^\tr_{\omega} \mathbf{F}_{AA}^\prime, \mathbf{1}^\tr_{\omega} \mathbf{F}_{AA}^\prime , \frac{1}{2}\mathbf{1}^\tr_{\omega} \mathbf{F}_{Aa}^\prime, \frac{1}{2}\mathbf{1}^\tr_{\omega} \mathbf{F}_{Aa}^\prime, \mathbf{0}, \mathbf{0} \right)\label{eq:first_term_app_reviewer}
\end{eqnarray}
Similarly for the second term in the sum in equation (\ref{eq:56_app}),
\begin{eqnarray}
	- \frac{\mathbf{e}_1^\tr \mathbf{W}^\prime \mathbf{F}^\prime \mathbf{p}^\prime}{ \mbox{p}_{n}^2} \left( \mathbf{1}^\tr_{2}\mathbf{W}^\prime\mathbf{F}^\prime \frac{\partial  \mathbf{p}^\prime}{\partial \tilde{\bo p}^\tr} \right) &=& -\frac{1}{ \mbox{p}_{n}}(1,1) \left(\begin{array} {ccc}
\mathbf{1}^\tr_{\omega} \mathbf{F}_{AA}^\prime& \frac{1}{2}\mathbf{1}^\tr_{\omega} \mathbf{F}_{Aa}^\prime &\mathbf{0} \\
\mathbf{0} &\frac{1}{2}\mathbf{1}^\tr_{\omega} \mathbf{F}_{Aa}^\prime & \mathbf{1}^\tr_{\omega} \mathbf{F}_{aa}^\prime
\end{array}\right) \left(\begin{array} {cccccc}
\mathbf{I}& \mathbf{I}& \mathbf{0}  &\mathbf{0} &\mathbf{0}  &\mathbf{0} \\
\mathbf{0} &\mathbf{0}  &\mathbf{I}& \mathbf{I}& \mathbf{0} &\mathbf{0} \\
\mathbf{0} &\mathbf{0}  &\mathbf{0} &\mathbf{0}  &  \mathbf{I}& \mathbf{I}
\end{array}\right) \nonumber \\
 &=& -\frac{1}{ \mbox{p}_{n}} \left( \mathbf{1}^\tr_{\omega} \mathbf{F}_{AA}^\prime, \mathbf{1}^\tr_{\omega} \mathbf{F}_{AA}^\prime , \mathbf{1}^\tr_{\omega} \mathbf{F}_{Aa}^\prime, \mathbf{1}^\tr_{\omega} \mathbf{F}_{Aa}^\prime, \mathbf{1}^\tr_{\omega} \mathbf{F}_{aa}^\prime , \mathbf{1}^\tr_{\omega} \mathbf{F}_{aa}^\prime  \right) \label{eq:second_term_app_reviewer}
\end{eqnarray}
Finally add equations (\ref{eq:first_term_app_reviewer}) and (\ref{eq:second_term_app_reviewer}) to obtain
\begin{eqnarray}
  \frac{\partial   q^\prime_A }{\partial \tilde{\bo p}^\tr}\bigg\rvert_{\hat{\bo p}}&=& \frac{1}{ \mbox{p}_{n}} \left( \mathbf{0}, \mathbf{0}, -\frac{1}{2}\mathbf{1}^\tr_{\omega} \mathbf{F}_{Aa}^\prime, -\frac{1}{2}\mathbf{1}^\tr_{\omega} \mathbf{F}_{Aa}^\prime, -\mathbf{1}^\tr_{\omega} \mathbf{F}_{aa}^\prime, -\mathbf{1}^\tr_{\omega} \mathbf{F}_{aa}^\prime  \right) \label{eq:79_app},
\end{eqnarray}
where at the boundary
\begin{eqnarray}
  \mbox{p}_{n}&=&\mathbf{1}^\tr_\omega \mathbf{F}_{AA}^\prime \left( \hat{\mathbf{p}}^S_{AA}+\hat{\mathbf{p}}^X_{AA} \right), \\
  \mathbf{e}_1^\tr \mathbf{W}^\prime \mathbf{F}^\prime \mathbf{p}^\prime&=&\mathbf{1}^\tr_\omega \mathbf{F}_{AA}^\prime \left( \hat{\mathbf{p}}^S_{AA}+\hat{\mathbf{p}}^X_{AA} \right)=  \mbox{p}_{n}
\end{eqnarray}

\begin{landscape}
Finally, plugging equation (\ref{eq:79_app}) into (\ref{eq:partCwithpartials_app}) yields

{
\scriptsize
\begin{align*}
	\circled{C} =& \\
	\frac{(1 - C_{AA})}{\mbox{p}_{n}\lambda_{AA}} &\left(
			\begin{array}{cc|cc|cc}
					\mbf{0} & \mbf{0} & \mbf{0} & \mbf{0} & \mbf{0} & \mbf{0}  \\
				\mbf{0}   &  	\mbf{0}  & -\frac{1}{2} \mbf{F}_{AA} \big(\hat{\mbf{p}}_{X,AA}+ \hat{\mbf{p}}_{S,AA}  \big) \otimes \mathbf{1}^\tr_{\omega} \mathbf{F}_{Aa}^\prime& -\frac{1}{2} \mbf{F}_{AA} \big(\hat{\mbf{p}}_{X,AA}+ \hat{\mbf{p}}_{S,AA}  \big) \otimes \mathbf{1}^\tr_{\omega} \mathbf{F}_{Aa}^\prime & - \mbf{F}_{AA} \big(\hat{\mbf{p}}_{X,AA}+ \hat{\mbf{p}}_{S,AA}  \big) \otimes \mathbf{1}^\tr_{\omega} \mathbf{F}_{aa}^\prime&  - \mbf{F}_{AA} \big(\hat{\mbf{p}}_{X,AA}+ \hat{\mbf{p}}_{S,AA}  \big) \otimes \mathbf{1}^\tr_{\omega} \mathbf{F}_{aa}^\prime \\ \hline
				\mbf{0} & \mbf{0} & \mbf{0} & \mbf{0} & \mbf{0} & \mbf{0} \\
					\mbf{0}   &  	\mbf{0}  & \frac{1}{2} \mbf{F}_{AA} \big(\hat{\mbf{p}}_{X,AA}+ \hat{\mbf{p}}_{S,AA}  \big) \otimes \mathbf{1}^\tr_{\omega} \mathbf{F}_{Aa}^\prime& \frac{1}{2} \mbf{F}_{AA} \big(\hat{\mbf{p}}_{X,AA}+ \hat{\mbf{p}}_{S,AA}  \big) \otimes \mathbf{1}^\tr_{\omega} \mathbf{F}_{Aa}^\prime &  \mbf{F}_{AA} \big(\hat{\mbf{p}}_{X,AA}+ \hat{\mbf{p}}_{S,AA}  \big) \otimes \mathbf{1}^\tr_{\omega} \mathbf{F}_{aa}^\prime&  \mbf{F}_{AA} \big(\hat{\mbf{p}}_{X,AA}+ \hat{\mbf{p}}_{S,AA}  \big) \otimes \mathbf{1}^\tr_{\omega} \mathbf{F}_{aa}^\prime \\  \hline
				\mbf{0} & \mbf{0} & \mbf{0} & \mbf{0} & \mbf{0} & \mbf{0} \\ 
				\mbf{0} & \mbf{0} & \mbf{0} & \mbf{0} & \mbf{0} & \mbf{0} 
			\end{array} \right)  \numberthis			
\end{align*}
}

\subsection{The Jacobian}

Putting all the pieces together into $M$, we get the Jacobian:

{
\footnotesize
\begin{align*} \label{eq:combinedM}
	\mbf{M} = 
			 +\frac{1}{\lambda_{AA}} &
						   \left(
			\begin{array}{cc|cc|cc}
		 \mathbf{U}^S_{AA} +C_{AA} (1 - \delta) \mbf{F}_{AA} & C_{AA} (1 - \delta) \mbf{F}_{AA}& \frac{1}{4} C_{Aa} (1 - \delta) \mbf{F}_{Aa} & \frac{1}{4} C_{Aa} (1 - \delta) \mbf{F}_{Aa}& \mbf{0} & \mbf{0} \\ 
				(1 - C_{AA}) \mbf{F}_{AA} &\mathbf{U}^X_{AA} + (1 - C_{AA}) \mbf{F}_{AA}  & \frac{1}{2}(1 - C_{Aa}) \mbf{F}_{Aa}  & \frac{1}{2} (1 - C_{Aa})\mbf{F}_{Aa} & \mbf{0} & \mbf{0} \\ \hline
				\mbf{0} & \mbf{0} & \mathbf{U}^S_{Aa} +\frac{1}{2} C_{Aa} (1 - \delta) \mbf{F}_{Aa} & \frac{1}{2} C_{Aa} (1 - \delta) \mbf{F}_{Aa}  & \mbf{0} & \mbf{0} \\
				\mbf{0} & \mbf{0} & \frac{1}{2} (1 - C_{Aa}) \mbf{F}_{Aa} &\mathbf{U}^X_{Aa} + \frac{1}{2} (1 - C_{Aa}) \mbf{F}_{Aa} & (1 - C_{aa}) \mbf{F}_{aa} & (1 - C_{aa}) \mbf{F}_{aa}\\ \hline
				\mbf{0} & \mbf{0} & \frac{1}{4} C_{Aa} (1 - \delta) \mbf{F}_{Aa} & \frac{1}{4} C_{Aa} (1 - \delta) \mbf{F}_{Aa}  &\mathbf{U}^S_{aa} +C_{aa} (1 - \delta) \mbf{F}_{aa} &C_{aa} (1 - \delta) \mbf{F}_{aa} \\ 
				\mbf{0} & \mbf{0} & \mbf{0}  & \mbf{0} & \mbf{0} &\mathbf{U}^X_{aa} 
			\end{array} \right) \\
		-\frac{1}{\lambda_{AA}} &\left(
			\begin{array}{cc|cc|cc}
				\hat{\mbf{p}}_{AA} \otimes \mbf{1}^{\intercal}_{\omega} \mbf{U}^S_{AA} & \hat{\mbf{p}}_{AA} \otimes \mbf{1}^{\intercal}_{\omega} \mbf{U}^X_{AA} & \hat{\mbf{p}}_{AA} \otimes \mbf{1}^{\intercal}_{\omega} \mbf{U}^S_{Aa} & \hat{\mbf{p}}_{AA} \otimes \mbf{1}^{\intercal}_{\omega} \mbf{U}^X_{Aa} & \hat{\mbf{p}}_{AA} \otimes \mbf{1}^{\intercal}_{\omega} \mbf{U}^S_{aa} & \hat{\mbf{p}}_{AA} \otimes \mbf{1}^{\intercal}_{\omega} \mbf{U}^X_{aa} \\ 
				\hat{\mbf{p}}_{AA} \otimes \mbf{1}^{\intercal}_{\omega} \mbf{U}^S_{AA} & \hat{\mbf{p}}_{AA} \otimes \mbf{1}^{\intercal}_{\omega} \mbf{U}^X_{AA} & \hat{\mbf{p}}_{AA} \otimes \mbf{1}^{\intercal}_{\omega} \mbf{U}^S_{Aa} & \hat{\mbf{p}}_{AA} \otimes \mbf{1}^{\intercal}_{\omega} \mbf{U}^X_{Aa} & \hat{\mbf{p}}_{AA} \otimes \mbf{1}^{\intercal}_{\omega} \mbf{U}^S_{aa} & \hat{\mbf{p}}_{AA} \otimes \mbf{1}^{\intercal}_{\omega} \mbf{U}^X_{aa} \\ \hline
				\mbf{0} & \mbf{0} & \mbf{0} & \mbf{0} & \mbf{0} & \mbf{0} \\
				\mbf{0} & \mbf{0} & \mbf{0} & \mbf{0} & \mbf{0} & \mbf{0} \\ \hline
				\mbf{0} & \mbf{0} & \mbf{0} & \mbf{0} & \mbf{0} & \mbf{0} \\ 
				\mbf{0} & \mbf{0} & \mbf{0} & \mbf{0} & \mbf{0} & \mbf{0} 
			\end{array} \right) \\
		-\frac{1}{\lambda_{AA}}& \left(
			\begin{array}{cc|cc|cc}
				(1 - \delta C_{AA}) \hat{\mbf{p}}_{AA} \otimes \mbf{1}^{\intercal}_{\omega} \mbf{F}_{AA} & (1 - \delta C_{AA}) \hat{\mbf{p}}_{AA} \otimes \mbf{1}^{\intercal}_{\omega} \mbf{F}_{AA} & (1 - \delta C_{Aa}) \hat{\mbf{p}}_{AA} \otimes \mbf{1}^{\intercal}_{\omega} \mbf{F}_{Aa} & (1 - \delta C_{Aa}) \hat{\mbf{p}}_{AA} \otimes \mbf{1}^{\intercal}_{\omega} \mbf{F}_{Aa}  & (1 - \delta C_{aa}) \hat{\mbf{p}}_{AA} \otimes \mbf{1}^{\intercal}_{\omega} \mbf{F}_{aa} & (1 - \delta C_{aa}) \hat{\mbf{p}}_{AA} \otimes \mbf{1}^{\intercal}_{\omega} \mbf{F}_{aa} \\ 
				 	(1 - \delta C_{AA}) \hat{\mbf{p}}_{AA} \otimes \mbf{1}^{\intercal}_{\omega} \mbf{F}_{AA} & 	(1 - \delta C_{AA}) \hat{\mbf{p}}_{AA} \otimes \mbf{1}^{\intercal}_{\omega} \mbf{F}_{AA} & (1 - \delta C_{Aa})  \hat{\mbf{p}}_{AA} \otimes \mbf{1}^{\intercal}_{\omega} \mbf{F}_{Aa} & (1 - \delta C_{Aa})  \hat{\mbf{p}}_{AA} \otimes \mbf{1}^{\intercal}_{\omega} \mbf{F}_{Aa} & (1 - \delta C_{aa}) \hat{\mbf{p}}_{AA} \otimes \mbf{1}^{\intercal}_{\omega} \mbf{F}_{aa} & (1 - \delta C_{aa}) \hat{\mbf{p}}_{AA} \otimes \mbf{1}^{\intercal}_{\omega} \mbf{F}_{aa}\\ \hline
				\mbf{0} & \mbf{0} & \mbf{0} & \mbf{0} & \mbf{0} & \mbf{0} \\
				\mbf{0} & \mbf{0} & \mbf{0} & \mbf{0} & \mbf{0} & \mbf{0} \\ \hline
				\mbf{0} & \mbf{0} & \mbf{0} & \mbf{0} & \mbf{0} & \mbf{0} \\ 
				\mbf{0} & \mbf{0} & \mbf{0} & \mbf{0} & \mbf{0} & \mbf{0}
			\end{array} \right)\\
		\frac{(1 - C_{AA})}{\mbox{p}_{n}\lambda_{AA}} &\left(
			\begin{array}{cc|cc|cc}
					\mbf{0} & \mbf{0} & \mbf{0} & \mbf{0} & \mbf{0} & \mbf{0}  \\
				\mbf{0}   &  	\mbf{0}  & -\frac{1}{2} \mbf{F}_{AA} \big(\hat{\mbf{p}}_{X,AA}+ \hat{\mbf{p}}_{S,AA}  \big) \otimes \mathbf{1}^\tr_{\omega} \mathbf{F}_{Aa}^\prime& -\frac{1}{2} \mbf{F}_{AA} \big(\hat{\mbf{p}}_{X,AA}+ \hat{\mbf{p}}_{S,AA}  \big) \otimes \mathbf{1}^\tr_{\omega} \mathbf{F}_{Aa}^\prime & - \mbf{F}_{AA} \big(\hat{\mbf{p}}_{X,AA}+ \hat{\mbf{p}}_{S,AA}  \big) \otimes \mathbf{1}^\tr_{\omega} \mathbf{F}_{aa}^\prime&  - \mbf{F}_{AA} \big(\hat{\mbf{p}}_{X,AA}+ \hat{\mbf{p}}_{S,AA}  \big) \otimes \mathbf{1}^\tr_{\omega} \mathbf{F}_{aa}^\prime \\ \hline
				\mbf{0} & \mbf{0} & \mbf{0} & \mbf{0} & \mbf{0} & \mbf{0} \\
					\mbf{0}   &  	\mbf{0}  & \frac{1}{2} \mbf{F}_{AA} \big(\hat{\mbf{p}}_{X,AA}+ \hat{\mbf{p}}_{S,AA}  \big) \otimes \mathbf{1}^\tr_{\omega} \mathbf{F}_{Aa}^\prime& \frac{1}{2} \mbf{F}_{AA} \big(\hat{\mbf{p}}_{X,AA}+ \hat{\mbf{p}}_{S,AA}  \big) \otimes \mathbf{1}^\tr_{\omega} \mathbf{F}_{Aa}^\prime &  \mbf{F}_{AA} \big(\hat{\mbf{p}}_{X,AA}+ \hat{\mbf{p}}_{S,AA}  \big) \otimes \mathbf{1}^\tr_{\omega} \mathbf{F}_{aa}^\prime&  \mbf{F}_{AA} \big(\hat{\mbf{p}}_{X,AA}+ \hat{\mbf{p}}_{S,AA}  \big) \otimes \mathbf{1}^\tr_{\omega} \mathbf{F}_{aa}^\prime \\  \hline
				\mbf{0} & \mbf{0} & \mbf{0} & \mbf{0} & \mbf{0} & \mbf{0} \\ 
				\mbf{0} & \mbf{0} & \mbf{0} & \mbf{0} & \mbf{0} & \mbf{0} 
			\end{array} \right) 
			\numberthis
\end{align*}
}

\end{landscape}



%%%%%%%%%%%%%%%
\subsection{Eigenvalues of the Jacobian}

The Jacobian matrix given in equation (\ref{eq:combinedM}) is upper block triangular, and the eigenvalues of $\mbf{M}$ are therefore the eigenvalues of the diagonal blocks. Because of the way that self-fertilization alters the projection matrix, our blocks are different than those in both  \cite{deVriesCaswell2019a} and \cite{deVriesCaswell2019b}. The largest absolute eigenvalue of the Jacobian, i.e. the spectral radius $\rho(\mathbf{M})$, determines the stability of the boundary equilibrium. We will denote the three blocks along the diagonal with $\mathbf{M}_{11}$, $\mathbf{M}_{22}$, and $\mathbf{M}_{33}$, such that 
\begin{equation}
\mathbf{M}_{33}=\frac{1}{\lambda_{AA}}\mathbf{U}^X_{aa} .
\end{equation}
Block $\mathbf{M}_{33}$ projects perturbations in the $aa$ outcrossing $aa$ direction. In the neighbourhood of the $AA$ equilibrium, $aa$ homozygotes are rare, and thus $\mathbf{M}_{33}$ normally does not determine the stability of $\mathbf{M}$. An exception occurs when 

\begin{equation}
  \lambda_{AA}<\rho\left(\mathbf{U}^X_{aa} \right)<1.
\end{equation}
 That is, if the $AA$ population is declining sufficiently rapidly, the $aa$ homozygote may increase in frequency simply by declining to extinction more slowly. If the homozygous $AA$ population is stable or increasing, so that $\lambda_{AA} \geq 1$, this cannot happen. Similarly, if $\mathbf{U}^X_{aa}$ is age-classified with a maximum age, $\rho(\mathbf{U}^X_{aa})=0$, and the phenomenon can not happen.  We neglect this pathological case in our discussions.  The first 2$\omega\times$2$\omega$ block on the diagonal, $\mathbf{M}_{11}$, projects perturbations in the $AA$ boundary, and because $\hat{\mathbf{p}}$ is stable to perturbations in that boundary, $\rho(\mathbf{M}_{11})<1$. 

The stability of $\hat{\mathbf{p}}$ is thus determined by the middle diagonal block, 
{
\footnotesize
 \begin{align*} \label{eq:M22}
	\mbf{M}_{22} = 
		\frac{1}{\lambda_{AA}}\left(\begin{array}{ccc}
\mathbf{U}^S_{Aa} +\frac{1}{2} C_{Aa} (1 - \delta) \mbf{F}_{Aa} & \frac{1}{2} C_{Aa} (1 - \delta) \mbf{F}_{Aa}  & 0 \\
 \frac{1}{2} (1 - C_{Aa}) \mbf{F}_{Aa}+\frac{1}{2}  \Theta_{Aa} &\mathbf{U}^X_{Aa} + \frac{1}{2} (1 - C_{Aa}) \mbf{F}_{Aa} +\frac{1}{2}\Theta_{Aa} & (1-C_{aa})\mbf{F}_{aa} + \Theta_{aa}\\
  \frac{1}{4} C_{Aa} (1 - \delta) \mbf{F}_{Aa} & \frac{1}{4} C_{Aa} (1 - \delta) \mbf{F}_{Aa}  &\mathbf{U}^S_{aa} +C_{aa} (1 - \delta) \mbf{F}_{aa}  \\ 
			\end{array} \right), \numberthis
\end{align*}
}
where $\Theta_{aa} = \frac{(1 - C_{AA})}{p_{n}} \mbf{F}_{AA} \big(\hat{\mbf{p}}_{X,AA}+ \hat{\mbf{p}}_{S,AA}  \big) \otimes \mathbf{1}^\tr_{\omega} \mathbf{F}_{aa}^\prime$, and  $\Theta_{Aa} = \frac{(1 - C_{AA})}{p_{n}} \mbf{F}_{AA} \big(\hat{\mbf{p}}_{X,AA}+ \hat{\mbf{p}}_{S,AA}  \big) \otimes \mathbf{1}^\tr_{\omega} \mathbf{F}_{Aa}^\prime$

\begin{equation}
  \mbox{p}_{n}=\mathbf{1}^\tr_\omega \mathbf{F}_{AA}^\prime \left( \hat{\mathbf{p}}^S_{AA}+\hat{\mathbf{p}}^X_{AA} \right)	
\end{equation}
Note that the outcrossing terms are the sum of a term due to heterozygote ovules being fertilized by pollen of the resident ($\frac{1}{2} (1 - C_{Aa}) \mbf{F}_{Aa}$), and heterozygote pollen fertilizing resident females ($\frac{1}{2 }\Theta_{Aa}=\frac{1}{2  \mbox{p}_{n}} (1 - C_{AA})\mbf{F}_{AA} \big(\hat{\mbf{p}}_{X,AA}+ \hat{\mbf{p}}_{S,AA}  \big) \otimes \mathbf{1}^\tr_{\omega} \mathbf{F}_{Aa}^\prime$.

 The largest absolute value of the eigenvalues of the Jacobian matrix, the leading eigenvalue, evaluated at the $AA$ boundary, denoted by $\tilde{\zeta}_{AA}$, is therefore
{
\footnotesize
\begin{align*} \label{eq:eigAAFull}
	\tilde{\zeta}_{AA} = 
		\frac{1}{\lambda_{AA}}\rho\left(\begin{array}{ccc}
\mathbf{U}^S_{Aa} +\frac{1}{2} C_{Aa} (1 - \delta) \mbf{F}_{Aa} & \frac{1}{2} C_{Aa} (1 - \delta) \mbf{F}_{Aa}  & 0 \\
 \frac{1}{2} (1 - C_{Aa}) \mbf{F}_{Aa}+\frac{1}{2}  \Theta_{Aa} &\mathbf{U}^X_{Aa} + \frac{1}{2} (1 - C_{Aa}) \mbf{F}_{Aa} +\frac{1}{2}\Theta_{Aa} & (1-C_{aa})\mbf{F}_{aa} + \Theta_{aa}\\
  \frac{1}{4} C_{Aa} (1 - \delta) \mbf{F}_{Aa} & \frac{1}{4} C_{Aa} (1 - \delta) \mbf{F}_{Aa}  &\mathbf{U}^S_{aa} +C_{aa} (1 - \delta) \mbf{F}_{aa}  \\ 
			\end{array} \right). \numberthis
\end{align*}
}

% \noindent Rearranging the projection matrix so that outcrossing comes on top of selfing in each block simplifies things, but in the end the leading eigenvalue evaluated at the $aa$ boundary is 

% \begin{align*} \label{eq:eigaa}
% 	\tilde{\zeta}_{aa} = 
% 		\frac{1}{\lambda_{AA}} \rho&\left(\begin{array}{c|cc}
% 				\mcal{U}^S_{AA} + C_{AA} (1 - \delta) \mbf{F}_{AA} & \mbf{0} & \frac{1}{4} C_{Aa} (1 - \delta) \mbf{F}_{Aa}\\ \hline
% 				\mbf{F}_{AA} (1 - C_{AA}) & \mcal{U}^X_{Aa} + \frac{(1 - C_{Aa})}{2 p_n} (\mbf{F}_{aa} \hat{\mbf{p}}_{aa}) \otimes \mbf{1}^{\intercal}_{\omega} \mbf{F}^{\prime}_{Aa} & \frac{1}{2} \mbf{F}_{Aa} (1 - C_{Aa}) \\ 
% 				\mbf{0} & -C_{Aa}(1 - \delta) (\mbf{F}_{aa} \hat{\mbf{p}}_{aa}) & \mcal{U}^S_{Aa} + \frac{1}{2} C_{Aa} (1 - \delta) \mbf{F}_{Aa} \\
% 			\end{array} \right). \numberthis
% \end{align*}

A boundary equilibrium is unstable to invasion by the rare allele if the leading eigenvalue of the Jacobian evaluated at the equilibrium is greater than $1$. The conditions for a protected polymorphism is that both boundaries are unstable (i.e., both $\tilde{\zeta}_{AA} > 1$ and $\tilde{\zeta}_{aa} > 1$). The conditions for coexistence are, therefore, as given in equation \textcolor{red}{(B1)} and \textcolor{red}{(B2)},


{\footnotesize
\begin{equation} \label{eq:coexist_AA}
	\lambda_{AA} < 
			\rho\left(\begin{array}{ccc}
\mathbf{U}^S_{Aa} +\frac{1}{2} C_{Aa} (1 - \delta) \mbf{F}_{Aa} & \frac{1}{2} C_{Aa} (1 - \delta) \mbf{F}_{Aa}  & 0 \\
 \frac{1}{2} (1 - C_{Aa}) \mbf{F}_{Aa}+\frac{1}{2}  \Theta_{Aa} &\mathbf{U}^X_{Aa} + \frac{1}{2} (1 - C_{Aa}) \mbf{F}_{Aa} +\frac{1}{2}\Theta_{Aa} & (1-C_{aa})\mbf{F}_{aa} + \Theta_{aa}\\
  \frac{1}{4} C_{Aa} (1 - \delta) \mbf{F}_{Aa} & \frac{1}{4} C_{Aa} (1 - \delta) \mbf{F}_{Aa}  &\mathbf{U}^S_{aa} +C_{aa} (1 - \delta) \mbf{F}_{aa}  \\ 
			\end{array} \right), \\ 
\end{equation} 


\begin{equation} \label{eq:coexist_aa}
	\lambda_{aa} < 
			\rho\left(\begin{array}{ccc}
\mathbf{U}^S_{Aa} +\frac{1}{2} C_{Aa} (1 - \delta) \mbf{F}_{Aa} & \frac{1}{2} C_{Aa} (1 - \delta) \mbf{F}_{Aa}  & 0 \\
 \frac{1}{2} (1 - C_{Aa}) \mbf{F}_{Aa}+\frac{1}{2 } \Phi_{Aa} &\mathbf{U}^X_{Aa} + \frac{1}{2} (1 - C_{Aa}) \mbf{F}_{Aa} +\frac{1}{2} \Phi_{Aa} & (1-C_{AA})\mbf{F}_{AA} + \Phi_{AA}\\
  \frac{1}{4} C_{Aa} (1 - \delta) \mbf{F}_{Aa} & \frac{1}{4} C_{Aa} (1 - \delta) \mbf{F}_{Aa}  &\mathbf{U}^S_{AA} + C_{AA} (1 - \delta) \mbf{F}_{AA}  \\ 
			\end{array} \right), \\ 
\end{equation} 
}
where $\Phi_{AA} = \frac{(1 - C_{aa})}{p_{n}} \mbf{F}_{aa} \big(\hat{\mbf{p}}_{X,aa}+ \hat{\mbf{p}}_{S,aa}  \big) \otimes \mathbf{1}^\tr_{\omega} \mathbf{F}_{AA}^\prime$  and $\Phi_{Aa}=\frac{(1 - C_{aa})}{p_{n}} \mbf{F}_{aa} \big(\hat{\mbf{p}}_{X,aa}+ \hat{\mbf{p}}_{S,aa}  \big) \otimes \mathbf{1}^\tr_{\omega} \mathbf{F}_{Aa}^\prime$.

Because these coexistence conditions contain the stable stage specific frequencies on the boundaries, $\hat{\mbf{p}}_{X,AA}$, and $\hat{\mbf{p}}_{S,aa}$, finding an analytical expression in terms of the model parameters (i.e., a closed form solution),  is in general a non trivial task. We therefore used numerical techniques as described in the main text to identify invasion and extinction thresholds, which could then be used to define regions of demographically viable polymorphism across $s_f \times s_m$ parameter space. All computer code necessary to reproduce the results are available at \url{https://github.com/colin-olito/SA-Hermaphrodites-wDemography}.


%%%%%%%%%%%%%%%%%%%%%
% Bibliography
%%%%%%%%%%%%%%%%%%%%%
%
% Any references in this section should also be copied into the main 
% template, under the heading
% \section*{References Cited Only in the Online Enhancements}
% except when they are already cited in the main text.
\bibliography{suppRefs.bib}


\end{document}
