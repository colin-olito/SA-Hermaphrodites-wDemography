\documentclass[11pt]{article}
% Preamble
\usepackage[sc]{mathpazo} %Like Palatino with extensive math support
\usepackage{fullpage}
\usepackage[authoryear,sectionbib,sort]{natbib}
\bibliographystyle{amnat}
\setlength{\bibsep}{0.0pt}
\linespread{1.7}
\usepackage[utf8]{inputenc}
\usepackage[left]{lineno}
\usepackage{titlesec}
\usepackage{amsmath}
\usepackage{amsfonts}
\usepackage{amssymb}
\usepackage[utf8]{inputenc}
\usepackage{color,soul}
\usepackage{booktabs}
\usepackage{tikz}
\usepackage{pdflscape}
\usepackage[colorlinks=true, allcolors=black]{hyperref}

% Section Header Formats
\titleformat{\section}[block]{\Large\bfseries\filcenter}{\thesection}{1em}{}
\titleformat{\subsection}[block]{\Large\itshape\filcenter}{\thesubsection}{1em}{}
\titleformat{\subsubsection}[block]{\large\itshape}{\thesubsubsection}{1em}{}
\titleformat{\paragraph}[runin]{\itshape}{\theparagraph}{1em}{}[. ]\renewcommand{\refname}{Literature Cited}

% Special Math Characters
\newcommand\encircle[1]{%
  \tikz[baseline=(X.base)] 
    \node (X) [draw, shape=circle, inner sep=0] {\strut #1};}


% Graphics package
\usepackage{graphicx}
\graphicspath{{../output/figs/}.pdf}

% Change default margins
\usepackage[top=0.75in, bottom=0.75in, left=0.75in, right=0.75in]{geometry}

% Definitions
\def\mathbi#1{\textbf{\em #1}}
\def\mbf#1{\mathbf{#1}}
\def\mbb#1{\mathbb{#1}}
\def\mcal#1{\mathcal{#1}}
\newcommand{\bo}[1]{{\bf #1}}
\newcommand{\tr}{{\mbox{\tiny \sf T}}}
\newcommand{\bm}[1]{\mbox{\boldmath $#1$}}

 \def\linenumberfont{\normalfont\scriptsize}
%===========================================
% Basic info from Journal

%%%%%%%%%%%%%%%%
% Line numbering
%%%%%%%%%%%%%%%%

% Please use line numbering with your initial submission and
% subsequent revisions. After acceptance, please turn line numbering
% off by adding percent signs to the lines %\usepackage{lineno} and
% to %\linenumbers{} and %\modulolinenumbers[3] below.
%
% To avoid line numbering being thrown off around math environments,
% the math environments have to be wrapped using
% \begin{linenomath*} and \end{linenomath*}
%
% (Thanks to Vlastimil Krivan for pointing this out to us!)

%%%%%%%%%%%%
% Authorship
%%%%%%%%%%%%
% Please remove authorship information while your paper is under review,
% unless you wish to waive your anonymity under double-blind review. You
% will need to add this information back in to your final files after
% acceptance.

% The journal does not have numbered sections in the main portion of
% articles. Please refrain from using section references (à la
% section~\ref{section:CountingOwlEggs}), and refer to sections by name
% (e.g. section ``Counting Owl Eggs'').

% You may wish to remove the Acknowledgments section while your paper 
% is under review (unless you wish to waive your anonymity under
% double-blind review) if the Acknowledgments reveal your identity.
% If you remove this section, you will need to add it back in to your
% final files after acceptance.

%If you have deposited data to Dryad, you should cite them somewhere in the main text (usually in the Methods or Results sections). A sentence like the following will do. All data are available in the Dryad Digital Repository (\citealt{CookEtAl2015}).

%===========================================

% This version of the LaTeX template was last updated on
% November 8, 2019.

\begin{document}
\title{Evolutionary demography and the maintenance of sexually antagonistic polymorphism in outcrossers and simultaneous hermaphrodites}
\author{Colin Olito$^{1,\ast}$ \\ 
Charlotte de Vries$^{2}$}
\date{\today}
\maketitle

\noindent{} 1. Department of Biology, Lund University, Lund 223 62, Sweden;

\noindent{} 2.  Department of Evolutionary Biology and Environmental Studies, University of Zurich, Zurich CH-8057, Switzerland;

\noindent{} $\ast$ Corresponding author; e-mail: colin.olito@gmail.com

\bigskip

\textit{Manuscript elements}: Figure~1, figure~2, table~1, online appendices~A and B (including figure~A1 and figure~A2). Figure~2 is to print in color.

\bigskip

\textit{Keywords}: Intralocus sexual conflict, Evolutionary Demography, Balancing selection, Hermaphrodite, mixed mating systems, inbreeding depression 

\bigskip

\textit{Manuscript type}: Article. %Or e-article, note, e-note, natural history miscellany, e-natural history miscellany, comment, reply, invited symposium, or historical perspective.

\bigskip

\noindent{\footnotesize Prepared using the suggested \LaTeX{} template for \textit{Am.\ Nat.}}

\linenumbers{}
\modulolinenumbers[3]

\newpage{}


%====================
% Begin Main Text
%====================

\section*{Outline}

OK, just putting some ideas out here based roughly on our conversations about an {\itshape AmNat}-style article, the results we have so far, and what I think we should be able to do with data from COMPADRE. 

\begin{enumerate}
	\item {\bf Introduction}
	\begin{enumerate}
		\item Briefly introduce sexual antagonism and intra-locus sexual conflict. (Colin)
		\item Simultaneous hermaphrodites, mixed mating systems, unique predictions for SA polymorphism. (Colin)
		\item Introduce Evolutionary Demography, pivot to focus on the consequences of taking demography into account explicitly for the maintenance of SA polymorphism. (Lotte)
		\item Introduce key difference between existing population genetic models of SA polymorphism and our models: possibility of diverse population and evolutionary dynamics. (Lotte then Colin)
		\item Explain and highlight "demographically viable" parameter space, indicate that we are interested in whether parameter space predicted to be polymorphic by Pop.~gen.~models is, indeed, "demographically viable" (Lotte then Colin)
		\begin{enumerate}
			\item Pop. Gen. models tacitly assume stable population size, parameterize genotypic fitnesses relative to the 'best performing' genotype. Non-overlapping generations. Casually define fitness as lifetime reproductive fitness of a given genotype.
			\item "In practice, these genotypic fitnesses emerge from myriad process acting throughout the life-history of individuals"
			\item these processes contribute to key demographic rates such as the intrinsic population growth rate, 
			\item 
		\end{enumerate}
		\item Building upon the two-sex stage-structured matrix models developed by \citet{deVriesCaswell2019a,deVriesCaswell2019b}, we develop stage-structured matrix models that can accommodate selection through male and female function in simultaneous hermaphrodites with mixed mating systems. The models also allow for the effects of inbreeding depression to manifest in several demographic parameters, allowing us to explore the fitness consequences of early- vs.~late-acting inbreeding depression. (Colin then Lotte)
	\end{enumerate}

	\item {\bf Methods}
		\begin{enumerate}
			\item Citation heavy review of de Vries \& Caswell AmNat (2018) \& de Vries \& Caswell AmNat (2019) TheorPopBiol
			\item Briefly describe extension to simultaneous hermaphrodites, citing Jordan \& Connallon (2014), Abbot \& Tazzyman (2014), Olito (2017). 
			\item Breakdown into individuals produced by selfing vs.~outcrossing.
			\item Brief anatomy of $\mathbf{\tilde{A}}$.
			\item Analysis. Invasion analysis \& abridged derivation of invasion conditions \& identification of extinction threholds.
			\item Cite Appendix with full derivation distilled from notes file.
		\end{enumerate}

	\item {\bf When is polymorphism demographically viable?}
	\begin{enumerate}
		\item Review previous predictions for outcrossers \& partial selfing
		\begin{enumerate}
			\item Greater scope for SA polymorphism in outcrossers vs.~selfers
			\item Greater scope for SA polymorphism when SA fitness effects are recessive (dominance reversal).
			\item but our model allows us to see where/when invasion of SA alleles can drive the population extinct
			\item \textbf{Fig.1 - illustration of funnel plots w/ extinction isoclines}
		\end{enumerate}

		\item Key results summarized in \textbf{Fig.~2}: 
		\begin{enumerate}
			\item For simplicity, temporarily assume no inbreeding depression ($\delta = 0$).
			\item When populations do not have high growth-rates (i.e., when $\lambda \approx 1$), much of of polymorphic parameter space becomes demographically inviable.
			\item Specifically, would-be polymorphic paramter space becomes inviable when segregating male-beneficial/female-deleterious alleles can reduce fecundity enough to drive the population extinct.
			\item This effect is most dramatic for separate-sexed species and predominant outcrossers. Selfing shelters the population from the female-deleterious fitness effects because there is no opportunity for selection via male function. This can also be thought of as reproductive assurance in our model....
			\item Convergence to pop.~gen.~predictions when $f$ is big.
		\end{enumerate}
	\end{enumerate}

	\item {\bf Effects of inbreeding depression}
	\begin{enumerate}
		\item Review effect of I.D.~in population genetic models: causes selfing populations to look more like outcrossing ones (at least in terms of where SA polymorphism is maintained). This is because a fraction of selfed ovules are aborted, meaning the overall ratio of outcrossed vs.~selfed progeny increases.
		\item Matrix model allows us to explore effects of I.D. in different parts of the life-cycle (early- vs.~late-acting I.D.). Point out that previous pop.~gen.~models only address early-acting I.D. via ovule viability.
		\item MAKE CLEAR that purging of deleterious recessives thought to cause I.D. is already included in the model results presented in Fig.~3
		\item Key results summarized in Fig.~3
		\begin{enumerate}
			\item Pop.~Gen.~models predict that higher early-acting ID should lead to greater proportion of polymorphic parameter space, while our models predict that we get a critical threshold of selfing at which population viability plummets.
			\item When and how I.D. influences fitness during the life-cyclecan change the location and steepness of the decline in polymorphic parameter space.
			\item Critically, largest deviations from pop.~gen.~predictions happen for predominantly selfing populations.
			\item A perhaps obvious, but sometimes overlooked implication of our model predictions: selfing can offer reproductive assurance, and SA polymorphisms can be maintained in a non-trivial proportion of parameter space in partially selfing populations... BUT demographically, the population must be able to "afford" the loss of selfed ovules/offspring.
			\item HIGHLIGHT that this result holds even though we incorporated decrease of the mutation load due to inbreeding (and corresponding decrease in I.D.)...	
		\end{enumerate}
	\end{enumerate}

	\item {\bf Mimulus as a potential real world example}
	\begin{enumerate}
		\item We can parameterize our model with real populations' demographic rates, and make explicit predictions regarding the opportunity for SA polymorphism given the species' demography. We provide two illustrative examples:
		\item Opportunities for SA polymorphism in {\itshape Mimulus guttatus} (now {\itshape Erythranthe guttata})
		\item Possibly use data from \citet{PetersonEtAl2016} Eagle Meadows population, which is the 'local' population in their big common garden local adaptation expeirment.
		\begin{enumerate}
			\item Illustrate funnel plot using {\itshape E.~gutatta} demographic rates.
			\item \textbf{Fig.~4}: PLOT inversion from \citet{LeeKelly2015}, on that funnel!
		\end{enumerate}

	\end{enumerate}

	\item Discussion
		\begin{enumerate}
			\item Classic prediction from demography that most species should have a $\lambda \approx 1$
			\item If true, the demographic impacts on the opportunities for the maintenance of sexually antagonistic polymorphisms may be pervasive...
			\item Condition dependence, fecundity, scope for sexual confict/antagonistic polymorphism?
			\item Do sexually dimorphic species exhibit demographic rates that increase opportunities for SA polymorphism?

		\end{enumerate}

\end{enumerate}





%%%%%%%%%%%%%%%%%%%%
\newpage{}
\section*{Abstract}
%%%%%%%%%%%%%%%%%%%%
\ldots 

\newpage{}

%%%%%%%%%%%%%%%%%%%%%%%%
\section*{Introduction}
%%%%%%%%%%%%%%%%%%%%%%%%

Whether a population persists or goes extinct depends in large part on its ability to adapt to the environment. Yet, the process of adaptation can be impeded by a variety of genetic and environmental factors, including deleterious mutations \citep{Haldane1957}, environmental shifts causing a population to chase a new phenotypic optimum \citep{Maynard-Smith1976, LandeShannon1996,OrrUnckless2008}, maladaptive gene-flow \citep{KirkpatrickBarton1997, BolnickNosil2007}, as well as genetic constraints arising from intrinsic features of the inheritance system \citep{ConnallonHall2018}. 

Conflicting selection and gene-flow between different classes of individuals, or genetic trade-offs between fitness components, can impose constraints on adaptation with particularly interesting evolutionary consequences \citep{CharlesworthHughes2000, ConnallonHall2018}. Such constraints hinder adaptation by preventing individuals (or classes of individuals) from reaching the phenotypic optimum in one or more fitness components, thereby increasing the population's overall extinction risk. At the same time, they provide an effective mechanism for the maintenance of genetic variation, which in turn influences the population's capactiy for future adaptation \citep{Fisher1930, CharlesworthHughes2000, ConnallonHall2018, MatthewsConnallon2019}. Hence, for genetically constrained traits the nature and extent of genetic variation observed in natural populations should reflect the balance between the contribution of constraints to the maintainance of genetic polymorphisms, and the demographic consequences of the resulting maladaptation.
%Genetic trade-offs between fitness components due to pleiotropy represent a form of genetic constraint with particularly interesting evolutionary consequences because they act as a double-edged sword
%The nature and extent of genetic variation observed in natural populations should therefore reflect the balance between the contribution of those constraints to the maintainance of genetic polymorphism and the demographic consequences of the resulting maladaptation.

Genetic trade-offs between male and female fitness are a common feature of sexually reproducing populations, and are thought to contribute not only to the maintenance of genetic variation, but also the evolution of sexual dimorphism and gender-related traits \citep{Lande1980, Rice1992, Charlesworth1999, RiceChippindale2001, BondurianskyChenoweth2009,Olito2019}. When such genetic trade-offs occur, "sexually antagonistic" selection (abbreviated SA hereafter) can arise when beneficial alleles for one sex are deleterious when expressed in the other \citep{Kidwell1977, Rice1992, ConnallonClark2012}. Moreover, alleles with opposing fitness effects through male and female sex functions create analogous genetic constraints on fitness in hermaphrodite populations, where both maternal and paternal reproductive success contribute jointly to each individuals' overall fitness \citep{LloydWebb1986, WebbLloyd1986, Abbott2011, JordanConnallon2014}. 

The evolutionary dynamics of SA alleles in hermaphrodites differ from those in dioecious populations because individuals may reproduce through a combination of self- and outcross fertilization \citep{Goodwillie2005, Igic2006, JarneAuld2006}. Such mixed-mating systems have important theoretical consequences for the maintenance of SA polymorphisms in hermaphrodite populations. Self-fertilization is predicted to reduce the total parameter space where balanced SA polymorphisms can be maintained, while simultaneously creating a bias in selection through the female sex function (\citealt{JordanConnallon2014}; but see \citealt{Tazzyman2015}. Yet, other processes, such as genetic linkage to other SA loci or a sex-determining region \citep{Otto2011, JordanCharlesworth2012, Olito2017, Olito2019}, or spatial heterogeneity and the complexity of the life-cycle \citep{Olito-etal-2018,ConnallonSharmaOlito2019} can expand the parameter space for SA polymorphism. Overall, both theoretical predictions and current empirical data suggest there is ample scope for SA trade-offs and the maintenance of SA polymorphisms in both dioecious and hermaphrodite populations \citep{Abbott2011, WangBarrett2020}, although identifying specific SA loci from genome sequence data remains challenging \citep{RuzickaESEB2020}.
\bigskip

\begin{enumerate}
	\item Introduce Evolutionary Demography, pivot to focus on the consequences of taking demography into account explicitly for the maintenance of SA polymorphism. [Lotte]
	\item Introduce key difference between existing population genetic models of SA polymorphism and our models: possibility of diverse population and evolutionary dynamics. [Lotte then Colin]
	\item Explain and highlight "demographically viable" parameter space, indicate that we are interested in whether parameter space predicted to be polymorphic by Pop.~gen.~models is, indeed, "demographically viable" [Lotte then Colin]
		\begin{enumerate}
			\item Pop. Gen. models tacitly assume large and/or stable population size, parameterize genotypic fitnesses relative to the 'best performing' genotype. Non-overlapping generations. Casually define fitness as lifetime reproductive fitness of a given genotype.
			\item "In practice, these genotypic fitnesses emerge from myriad process acting throughout the life-history of individuals"
			\item these processes contribute to key demographic rates such as the intrinsic population growth rate, 
		\end{enumerate}
\end{enumerate}
\bigskip

In this article, we build upon the models of \citet{deVriesCaswell2019a} and \citet{deVriesCaswell2019b} to develop a stage-structured mendelian matrix model that accommodates selection through male and female function in simultaneous hermaphrodites with mixed mating systems. Using the model, we jointly analyze the evolutionary and demographic consequences of SA selection in obligately outcrossing (equivalent to dioecious) and partially selfing hermaphrodite populations. We focus our analyses on identifying the parameter conditions under which SA polymorphisms are maintained by balancing selection {\itshape and} the population is also demographically viable (i.e., has a positive intrinsic growth rate). The model also allows for the effects of inbreeding depression to manifest in different life-history stages, enabling us to explore the fitness consequences of early- vs.~late-acting inbreeding depression on the model predictions.

A key finding from our analysis is that when the intrinsic population growth rate is close to $1$, the deleterious effects of segregating male-beneficial SA alleles on female fecundity can result in extinction over much of the parameter space for SA polymorphism. Furthermore, this demographically viable polymorphic parameter space is often biased towards 'female-beneficial/male-deleterious' SA alleles (i.e., those with weaker selection on the female- than male sex function). Under some conditions, self-fertilization can alleviate the demographic costs of balanced SA polymorphisms, but the concommitant effects of inbreeding depression greatly exacerbate them. Overall, our findings provide a more nuanced picture of the nature of SA genetic variation that we should expect to find in natural populations, where the fate of SA alleles and the populations harboring them is determined jointly by evolutionary and demographic processes.


%%%%%%%%%%%%%%%%%%%%%%%%
\section*{The Model}
%%%%%%%%%%%%%%%%%%%%%%%%

Here, we briefly derive a matrix model incorporating multiple life-cycle stages and a single bi-allelic locus under SA selection on male and female sex functions for a population of partially selfing simultaneous hermaphrodites. The derivation and some of the key results follow closely those presented in \citet{deVriesCaswell2019a} and \citet{deVriesCaswell2019b}; and in fact, the model presented here reduces to the two-sex stage-structured model of \citet{deVriesCaswell2019b} under obligate outcrossing. A full derivation of the model and analyses is presented in \hl{Appendix A}, and all simulation code necessary to reproduce the results are available at \url{https://github.com/colin-olito/SA-Hermaphrodites-wDemography}.

In the model, individuals are classified jointly by life-cycle stage ($1, \ldots, \omega$), genotype ($1, \ldots, g$), and whether they were produced by self- or outcross fertilization (denoted by $S$ and $X$ superscripts). Each genotype is characterized by a matrix of transition probabilities (including survival) and a matrix of reproductive output through male and female function. These matrices can include time variation or nonlinearities reflecting the environment or density dependence, but these will not be modelled here. Each stage contributes offspring to genotypes at the next time step according to transition matrices that are determined by the mating system and stage-structure of the population.

The population state at time $t$ is described by a stage $\times$ genotype distribution matrix, which can be transformed into a population state vector, $\tilde{\mbf{n}}(t)$, which is ordered by genotype, then stage, then by how individuals were produced (self vs.~outcrossing). For a single locus with two alleles ($A$ and $a$), we have genotypes $g \in \{AA,\, Aa,\, aa\}$, giving the population state vector:
\begin{linenomath*}
\begin{equation}
	\tilde{\mbf{n}}(t) =  \left[
								\begin{array}{c}
									\mbf{n}^{S}_{AA}(t) \\
									\mbf{n}^{S}_{Aa}(t) \\
									\mbf{n}^{S}_{aa}(t) \\ \hline
									\mbf{n}^{X}_{AA}(t) \\
									\mbf{n}^{X}_{Aa}(t) \\
									\mbf{n}^{X}_{aa}(t) \\ 
						\end{array} \right],
\end{equation}
\end{linenomath*}

\noindent where $\mbf{n}^{S}_{i}$ and $\mbf{n}^{X}_{i}$ (where $i \in \{AA,Aa,aa\}$) are the stage $\times$ genotype distribution vectors of individuals produced by self-fertilization and outcrossing, respectively. The proportional population vector is given by
\begin{linenomath*}
\begin{equation} \label{eq:propPopVec}
	\tilde{\mbf{p}}(t) = \frac{\tilde{\mbf{n}}(t)}{ \| \tilde{\mbf{n}}(t) \|} = \mbf{p}(t),
\end{equation}
\end{linenomath*}

\noindent where $\| \cdot \|$ is the one-norm. The population vector $\tilde{\mbf{n}}(t)$ is projected forward from time $t$ to $t + 1$ by the $projection$ matrix $\tilde{\mbf{A}}(\tilde{\mbf{n}})$ such that 
\begin{linenomath*}
\begin{equation}
	\tilde{\mbf{n}}(t + 1) = \tilde{\mbf{A}}[\tilde{\mbf{n}}(t)] \, \tilde{\mbf{n}}(t)
\end{equation}
\end{linenomath*}

The population projection matrix $\tilde{\mbf{A}}$ is constructed from four sets of matrices representing the demographic and genetic processes: The matrices $\mbf{U}^{S}_{i}$ and $\mbf{U}^{X}_{i}$ contains transition and survival probabilities for each genotype, produced by selfing and outcrossing respectively. The matrices $\mbf{F}_{i}$ and $\mbf{F}^{\prime}_{i}$ contain the genotype $\times$ stage specific contribution via female and male sex-functions of genotype $i$ to the female and male gamete pools, respectively, and therefore to zygotes in the next generation. In principle, the model can admit any kind of pleiotropy among demographic traits because the transition and fertility matrices can differ among genotypes in any way. For simplicity, we assume linear, time-invariant demography.

Similar to the two-sex model of \citet{deVriesCaswell2019b}, we define the matrices $\mbf{H}^S_{j}(\tilde{\mbf{n}})$ and $\mbf{H}^X_{j}(\tilde{\mbf{n}})$, which map the genotypes of the parent in stage $j$ to the genotypes of their offspring produced by selfing and outcrossing, respectively. The $(k, l)$ entry of $\mbf{H}^{X}_{j}$ and $\mbf{H}^{S}_{i}$ is the probability that an offspring of a genotype $l$ mother, of stage $i$, has genotype $k$. For the purpose of this article, we assume that mating is random with respect to stage and hence that the parent-offspring map is the same for all stages (i.e., $\mbf{H}^S_{j}(\tilde{\mbf{n}}) = \mbf{H}^S(\tilde{\mbf{n}})$, and $\mbf{H}^X_{j}(\tilde{\mbf{n}}) = \mbf{H}^X(\tilde{\mbf{n}})$). 

%%%%%%%%%%%%%%%%%%%%%%
\subsubsection*{Mating and offspring production under partial selfing}

The matrices $\mbf{H}^S(\tilde{\mbf{n}})$ and $\mbf{H}^X(\tilde{\mbf{n}})$ contain the population genetic processes and are derived in detail in \hl{Appendix A: Section A.x}. 	

%%%%%%%%%%%%%%%%%%%%%%
\subsection*{Analyses}

\ldots


\paragraph*{Fourth-order heading}
\ldots


%%%%%%%%%%%%%%%%%%%%%%%%
\section*{Results}
%%%%%%%%%%%%%%%%%%%%%%%%
\ldots



 \begin{figure}[htbp]
 \centering
 \includegraphics[width=\linewidth]{../output/figs/extinctionThresholdsFig.pdf}
 \caption{Illustration of parameter space for SA polymorphism and extinction thresholds predicted by the mendelian stage-structured matrix model. Balanced SA polymorphisms can be maintained in the funnel-shaped region between the invasion conditions for each SA allele (dark solid lines). However, for some parameter conditions, populations will ultimately go extinct (red shaded regions) due to reduced female fitness resulting from the male-beneficial/female-deleterious allele that is either segregating as a balanced polymorphism (inside the funnel), or becomes fixed (area below the funnel). We are particularly interested in identifying and quantifying "demographically viable polymorphic parameter space", which corresponds to the area inside the funnel that is also to the left of the extinction threshold for a given fertility value. Results are shown for three different population selfing rates ($C = \{0,\,1/4,\,1/2\}$), and two dominance scenarios (additivity with $h = 1/2$, and dominance reversal with $h = 1/4$); extinction thresholds are illustrated for three different values of female fecundity ($f$ values annotated on each panel). }
 \label{fig:extThresholds}
 \end{figure}

%%%%%%%%%%%%%%%%%%%%%%%%
\subsection*{The height of the jump}

\ldots

%%%%%%%%%%%%%%%%%%%%%%%%
\subsection*{The laziness of the dog}

\ldots

%%%%%%%%%%%%%%%%%%%%%%%%
\section*{Discussion}
%%%%%%%%%%%%%%%%%%%%%%%%

\ldots

%%%%%%%%%%%%%%%%%%%%%%%%
\section*{Conclusion}
%%%%%%%%%%%%%%%%%%%%%%%%

\ldots

%%%%%%%%%%%%%%%%%%%%%
% Acknowledgments
%%%%%%%%%%%%%%%%%%%%%

\section*{Acknowledgments}

OEC would like to thank Madlen Wilmes, Gyuri Barab\'{a}s, Flo D\'{e}barre, Vlastimil K\v{r}ivan, and Greg Dwyer for their comments and suggestions on this template.

\newpage{}

\section*{Appendix A: Supplementary Figures}

% In many cases, The American Naturalist allows authors to typeset 
% their own supplementary material in an author-supplied PDF. For author-
% supplied PDFs, please consult the AmNat_supp_template.tex document,
% available from https://www.journals.uchicago.edu/journals/an/instruct 
%
% By contrast, the Appendix instructions below apply to cases in which
% supplementary material is to be typeset by the AmNat editorial staff.
% That notably includes descriptions of methods, tables defining parameters,
% and other material necessary for reproducing the MS's results.
%
% Please reset counters for the appendix (thus normally figure A1, 
% figure A2, table A1, etc.).
%
% In certain cases, it may be appropriate to have a PRINT appendix in
% addition to (or instead of) an online appendix. In this case, please 
% name the print appendix Appendix A, and any subsequent appendixes (if 
% there are any) should be named Online Appendix B, Online Appendix C,
% etc.
%
% Counters for each appendix should match the letter of that appendix.
% For example, tables in Appendix C should be numbered table C1, table C2,
% etc. This applies to tables, equations, and figures.
%
% It's better not to use the \appendix command, because we have some
% formatting peculiarities that \appendix conflicts with.

\renewcommand{\theequation}{A\arabic{equation}}
\renewcommand{\thetable}{A\arabic{table}}
\setcounter{equation}{0}  % reset counter 
\setcounter{figure}{0}
\setcounter{table}{0}

\subsection*{Fox--dog encounters through the ages}

The quick red fox jumps over the lazy brown dog. The quick red fox has always jumped over the lazy brown dog. The quick red fox began jumping over the lazy brown dog in the 19th century and has never ceased from so jumping, as we shall see in figure~\ref{Fig:Jumps}. But there can be surprises (figure~\ref{Fig:JumpsOk}).

If the order and location of figures is not otherwise clear, feel free to include explanatory dummy text like this:

[Figure A1 goes here.]

[Figure A2 goes here.]

\subsection*{Further insights}

Tables in the appendices can appear in the appendix text (see table~\ref{Table:Rivers} for an example), unlike appendix figure legends which should be grouped at the end of the document together with the other figure legends.

\begin{table}[h]
\caption{Various rivers, cities, and animals}
\label{Table:Rivers}
\centering
\begin{tabular}{lll}\hline
River        & City        & Animal            \\ \hline
Chicago      & Chicago     & Raccoon           \\
Des Plaines  & Joliet      & Coyote            \\
Illinois     & Peoria      & Cardinal          \\
Kankakee     & Bourbonnais & White-tailed deer \\
Mississippi  & Galena      & Bald eagle        \\ \hline
\end{tabular}
\bigskip{}
\\
{\footnotesize Note: See table~\ref{Table:Founders} below for further table formatting hints.}
\end{table}

Lorem ipsum dolor sit amet, as we have seen in figures~\ref{Fig:Jumps} and \ref{Fig:JumpsOk}.

\newpage{}
\renewcommand{\theequation}{B\arabic{equation}}
% redefine the command that creates the equation number.
\renewcommand{\thetable}{B\arabic{table}}
\setcounter{equation}{0}  % reset counter 
\setcounter{table}{0}

\section*{Appendix B: Additional Methods}

\subsection*{Measuring the height of fox jumps without a meterstick}

Pellentesque ac nibh placerat, luctus lectus non, elementum mauris. 
Morbi odio velit, eleifend ut hendrerit vitae, consequat sit amet 
nulla. Pellentesque porttitor vitae nisl quis tempus. Pellentesque 
habitant morbi tristique senectus et netus et malesuada fames ac 
turpis egestas. Praesent ut nisi odio. Vivamus vel lorem gravida 
odio molestie volutpat condimentum et arcu. 

\begin{equation}
{ \frac{1}{N_k-1} \sum \limits_{t=1}^{N_k} (M_{tjk} - \bar{M}_{jk})^2}
\end{equation}

\subsection*{Quantifying the brownness of the dog}

Pellentesque eu nulla odio. Nulla aliquam porta metus, quis malesuada orci faucibus quis. Suspendisse nunc magna, tristique sit amet sollicitudin nec, elementum et lacus. Sed vitae elementum mi. In hac habitasse platea dictumst. Etiam eu tortor elit. Sed ac tortor purus. Aliquam volutpat, odio sit amet posuere pretium, dolor ex interdum ante, sed luctus quam eros ac nulla. 

\begin{equation}
{ (\sum \limits_{p=1}^P {n_{sp}})^{-1}\sum \limits_{p=1}^P {n_{sp}Q_{p}}}
\end{equation}

\newpage{}

%%%%%%%%%%%%%%%%%%%%%
% Bibliography
%%%%%%%%%%%%%%%%%%%%%
% You can either type your references following the examples below, or
% compile your BiBTeX database and paste the contents of your .bbl file
% here. The amnatnat.bst style file should work for this---but please
% let us know if you run into any hitches with it!
%
% If you upload a .bib file with your submission, please upload the .bbl
% file as well; this will be required for typesetting.
%
% The list below includes sample journal articles, book chapters, and
% Dryad references.
\bibliography{Refs2.bib}

\newpage{}

\section*{Tables}
\renewcommand{\thetable}{\arabic{table}}
\setcounter{table}{0}

\begin{table}[h]
\caption{Founders of \textit{The~American Naturalist}}
\label{Table:Founders}
\centering
\begin{tabular}{lll}\hline
Early editor            & Years with the journal \\ \hline
Alpheus S. Packard Jr.  & 1867--1886 \\
Frederick W. Putnam     & 1867--1874 \\ 
Edward S. Morse         & 1867--1871 \\ 
Alpheus Hyatt           & 1867--1871 \\
Edward Drinker Cope$^a$ & 1878--1897 \\
J.~S. Kingsley          & 1887--1896 \\ \hline 
\end{tabular}
\bigskip{}
\\
{\footnotesize Note: Table titles should be short. Further details should go in a `notes' area after the tabular environment, like this. $^a$ Published the first description of \textit{Dimetrodon}.}
\end{table}

\newpage{}

\section*{Figure legends}

\begin{figure}[h!]
%\includegraphics{horn-of-okapi}
\caption{Figure legends can be longer than the titles of tables. However, they should not be excessively long.}
\label{Fig:OkapiHorn}
\end{figure}


%%%%%%%%%%%%%%%%%%%%%
% Videos
%%%%%%%%%%%%%%%%%%%%%
% If you have videos, journal style for them is similar to that for
% figures. You'll want to include a still image (such as a JPEG)
% to give your readers a preview of what the video looks like.

%%%%% Include the text below if you have videos

\renewcommand{\figurename}{Video} 
\setcounter{figure}{0}
% Thanks to Flo Debarre for the pro tip of putting
% \renewcommand{\figurename}{Video} before the Video legend and
% \renewcommand{\figurename}{Figure} after it!

\begin{figure}[h!]
%\includegraphics{VideoScreengrab.jpg}
\caption{Video legends can follow the same principles as figure legends. Counters should be set and reset so that videos and figures are enumerated separately.}
\label{VideoExample}
\end{figure}

\renewcommand{\figurename}{Figure}
\setcounter{figure}{1}

%%%%% Include the above if you have videos


\begin{figure}[h!]
%\includegraphics{elegance}
\caption{In this way, figure legends can be listed at the end of the document, with references that work, even though the graphic itself should be included for final files after acceptance. Instead, upload the relevant figure files separately to Editorial Manager; Editorial Manager should insert them at the end of the PDF automatically.}
\label{Fig:AnotherFigure}
\end{figure}

\subsection*{Online figure legends}

\renewcommand{\thefigure}{A\arabic{figure}}
\setcounter{figure}{0}

\begin{figure}[h!]
%\includegraphics{jumps20m}
\caption{\textit{A}, the quick red fox proceeding to jump 20~m straight into the air over not one, but several lazy dogs. \textit{B}, the quick red fox landing gracefully despite the skepticism of naysayers.}
\label{Fig:Jumps}
\end{figure}

\begin{figure}[h!]
%\includegraphics{jumps20m}
\caption{The quicker the red fox jumps, the likelier it is to land near an okapi. For further details,.}
\label{Fig:JumpsOk}
\end{figure}

\renewcommand{\thefigure}{B\arabic{figure}}
\setcounter{figure}{0}

\end{document}
